% !TeX encoding = UTF-8
% !TeX spellcheck = en_US
% !TEX root = endterm_exam_WS2425.tex
\documentclass{article}
%%%%%%%%%%%%%%%%%%%%%%%%%%%%%%%%%%%%%%%%%%%%%%%%%%%%%%%%%%%%%%%%%%%%%%%%%%%%%%%%%%%%%%%%%%%%%%%%%%%%%%%%%%%%%%%%%%%%%%%%%%%%%%%%%%%%%%%%%%%%%%%%%%%%%%%%%%%%%%%%%%%%%%%%%%%%%%%%%%%%%%%%%%%%%%%%%%%%%%%%%%%%%%%%%%%%%%%%%%%%%%%%%%%%%%%%%%%%%%%%%%%%%%%%%%%%
\usepackage[a4paper,top=2cm]{geometry}
\usepackage{amssymb,amsmath,amsfonts}
\usepackage[english]{babel}
\usepackage[a4paper]{geometry}
\usepackage{enumitem}
\usepackage{booktabs}
\usepackage{csquotes}
\usepackage{graphicx}
\usepackage[numbered,framed]{matlab-prettifier}
\usepackage{url}
\usepackage[backend=biber,style=authoryear]{biblatex}
\addbibresource{../literature/_biblio.bib}
\begin{document}
	
\title{Quantitative Macroeconomics}
\author{Endterm Exam\\Version 1.0.1}
\date{Winter 2024/2025}

\maketitle

\section*{General information}
\begin{itemize}
	\item Answer \textbf{three} out of the following \textbf{four} exercises in English.
	\item All assignments will be given the same weight in the final grade.
	\item Hand in your solutions before February 23, 2025 at 3pm.
	\item The solution files should contain your executable (and commented) script files
		as well as all additional documentation as \texttt{pdf}, not \texttt{txt}, \texttt{md}, \texttt{tex}, \texttt{doc} or \texttt{docx}.
	Your \texttt{pdf} files may also include scans or pictures of handwritten notes.
	\item Please e-mail ALL the solution files to \url{willi.mutschler@uni-tuebingen.de}.
	I will confirm the receipt of your work also by email (typically within the hour). If not, please resend it to me.
	\item All students must work on their own, please also give your student ID number and the name of the module you want to earn credits for.
	\item It is advised to regularly check Ilias and your emails in case of urgent updates.
	\item If there are any questions, do not hesitate to contact Willi Mutschler.
\end{itemize}

\section*{Changelog}
\begin{itemize}
\item \textbf{Version 1.0.1:} Rearranged first equation in exercise 2.2, because the conditional expectation operator is not correctly applied \emph{mathematically}.
Old equation:
\begin{align*}
E_t\left[ \frac{C_{i,t+1}}{C_{i,t}} \right] = \beta E_t\left[1-\delta + R_{t+1}\right]
\end{align*}
Corrected equation:
\begin{align*}
1 = \beta E_t\left[ \frac{C_{i,t}}{C_{i,t+1}} \left(1-\delta + R_{t+1}\right)\right]
\end{align*}
This has no effect on the underlying codes, as Dynare automatically puts all terms on one side of the equation and then applies the conditional expectation operator.
\end{itemize}

\newpage

\section[Historical Decomposition]{Historical Decomposition}
Reconsider the SVAR model of \textcite{Rubio-Ramirez.Waggoner.Zha_2010_StructuralVectorAutoregressions}
  (i.e.\ the exercise called \emph{Combining Short-Run And Long-Run Restrictions}).

A historical decomposition is a tool to quantify how important a structural shock was
  in driving the behavior of the endogenous variables in a specific time period in the past.
It allows to track at each point in time the role of structural shocks
  in driving the VAR's endogenous variables away from their steady state.

Given the notation used in the lecture, the historical decomposition for a VAR{(1)}
  is based on the following recursive Wold representation:
\begin{align*}
y_t = A^t y_0 + \sum_{j=0}^{t-1} A^j B_0^{-1} \varepsilon_{t-j}
\end{align*}
Therefore, the historical decomposition of say \(y_{2}\) is given
  as a function of present and past structural shocks
  plus the initial condition:
\begin{align*}
y_{2} = \underbrace{A^2 y_0}_{\text{initial}} + \underbrace{A B_0^{-1}}_{\Theta_1} \varepsilon_{1} + \underbrace{B_0^{-1}}_{\Theta_0} \varepsilon_{2}
\end{align*}

\begin{enumerate}
\item Write an algorithm to compute the historical decomposition of the federal funds rate with respect to the three shocks in the model.
\item Interpret the results in economic terms.
\end{enumerate}

Hints:
\begin{itemize}
\item Use the companion form.
\end{itemize} 

\newpage

\section{RBC model with leisure and non-Ricardian agents}
Consider the basic Real Business Cycle (RBC) model with leisure and non-Ricardian agents.
Assume that there is a continuum of consumers given on the interval \([0,1]\).
A proportion of the population, \(\omega \), are Ricardian agents
  who have access to financial markets and are indexed by \(i \in [0,\omega]\).
The other part of the population, \(1-\omega \), is composed of non-Ricardian agents
  who do not have access to financial markets and are indexed by \(j \in [\omega,1]\).

A Ricardian household maximizes present as well as expected future utility
\begin{align*}
\underset{\{C_{i,t},I_{i,t},L_{i,t},K_{i,t}\}}{\max} E_t \sum_{s=0}^{\infty} \beta^{s} U_{i,t+s}
\end{align*}
with \(\beta <1\) denoting the discount factor and \(E_t\) is expectation given information at time \(t\).
The contemporaneous utility function is given by
\begin{align*}
U_{i,t} = \gamma \ln C_{i,t} + (1-\gamma) \ln{(1-L_{i,t})}
\end{align*}
and has two arguments: consumption \(C_{i,t}\) and labor \(L_{i,t}\).
The marginal utility of consumption is positive, whereas more labor reduces utility.
Accordingly, \(\gamma \) is the elasticity of substitution between consumption and labor.
In each period the household takes the real wage \(W_t\) as given
  and supplies perfectly elastic labor service to the representative firm.
In return, they receive real labor income in the amount of \(W_t L_{i,t}\)
  and, additionally, profits \(\Pi_{i,t}\) from the firm
  as well as revenue from lending capital in the previous period \(K_{i,t-1}\) at interest rate \(R_t\) to the firms,
  as it is assumed that the firm and capital stock are owned by the Ricardian households.
Income and wealth are used to finance consumption \(C_{i,t}\) and investment \(I_{i,t}\).
In total, this defines the (real) budget constraint of the Ricardian agent:
\begin{align*}
C_{i,t} + I_{i,t} = W_t L_{i,t} + R_t K_{i,t-1} + \Pi_{i,t}
\end{align*}

The law of motion for capital \(K_{i,t}\) at the end of period \(t\) is given by
\begin{align*}
K_{i,t} = (1-\delta)K_{i,t-1} + I_{i,t}
\end{align*}
where \(\delta \) is the depreciation rate.
Assume that the transversality condition is full-filled.

A non-Ricardian household maximizes present as well as expected future utility
\begin{align*}
\underset{\{C_{j,t},L_{j,t}\}}{\max} E_t \sum_{s=0}^{\infty} \beta^{s} U_{j,t+s}
\end{align*}
The contemporaneous utility function is the same as for non-Ricardian households, i.e.\
\begin{align*}
U_{j,t} = \gamma \ln C_{j,t} + (1-\gamma) \ln{(1-L_{j,t})}
\end{align*}
As non-Ricardian agents do not have access to the credit market, their (real) budget constraint is given by:
\begin{align*}
C_{j,t} = W_t L_{j,t}
\end{align*}

It is assumed that all agents, independently the group they belong to, are identical.
Therefore, aggregate values (in per capita terms) are given by:
\begin{align*}
C_t = \omega C_{i,t} + (1-\omega)C_{j,t},
&& L_t = \omega L_{i,t} + (1-\omega)L_{j,t},
&& K_t = \omega K_{i,t}
&& I_t = \omega I_{i,t}
\end{align*}
where the right two expressions are due to the fact that only Ricardian agents invest in physical capital.

Productivity \(A_t\) is the driving force of the economy and evolves according to
\begin{align*}
\ln{A_{t}} &= \rho_A \ln{A_{t-1}}  + \varepsilon_t^A
\end{align*}
where \(\rho_A\) denotes the persistence parameter and \(\varepsilon_t^A\) is assumed to be normally distributed with mean zero and variance \(\sigma_A^2\).

Real profits \(\Pi_t = \omega \Pi_{i,t}\) of the representative firm are revenues from selling output \(Y_t\) minus costs from labor \(W_t L_t\) and renting capital \(R_t K_{t-1}\):
\begin{align*}
\Pi_t = Y_{t} - W_{t} L_{t} - R_{t} K_{t-1}
\end{align*}
The representative firm maximizes expected profits
\begin{align*}
\underset{\{L_{t},K_{t-1}\}}{\max} E_t \sum_{j=0}^{\infty} \beta^j Q_{t+j}\Pi_{t+j}
\end{align*}
subject to a Cobb-Douglas production function
\begin{align*}
Y_t = A_t K_{t-1}^\alpha L_t^{1-\alpha}
\end{align*}
The discount factor takes into account that firms are owned by the Ricardian households,
  i.e.\ \(\beta^s Q_{t+s}\) is the present value of a unit of consumption in period \(t+s\)
  or, respectively, the marginal utility of an additional unit of profit;
  therefore \(Q_{t+s}=\frac{\partial U_{i,t+s}/\partial C_{i,t+s}}{\partial U_{i,t}/\partial C_{i,t}}\).

Finally, we have the non-negativity constraints
  \(K_{i,t} \geq0\), \(C_{i,t} \geq 0\), \(C_{j,t} \geq 0\), \(0\leq L_{i,t} \leq 1\) and \(0\leq L_{j,t} \leq 1\).
Furthermore, clearing of the labor as well as goods market in equilibrium implies
\begin{align*}
Y_t = C_t + I_t
\end{align*}

\begin{enumerate}

\item Briefly provide intuition behind the introduction of non-Ricardian households.

\item Show that the first-order conditions of the agents are given by
\begin{align*}
1 = \beta E_t\left[ \frac{C_{i,t}}{C_{i,t+1}} \left(1-\delta + R_{t+1}\right)\right],
&&
W_t = \frac{1-\gamma}{\gamma} \frac{C_{i,t}}{1-L_{i,t}},
\\
C_{j,t} = W_t L_{j,t},
&&
W_t = \frac{1-\gamma}{\gamma} \frac{C_{j,t}}{1-L_{j,t}}.
\end{align*}
Interpret these equations in economic terms.

\item Show that the first-order conditions of the representative firm are given by
\begin{align*}
W_t = (1-\alpha) A_t \left(\frac{K_{t-1}}{L_t}\right)^\alpha, &&	R_t = \alpha A_t \left(\frac{L_t}{K_{t-1}}\right)^{1-\alpha}
\end{align*}
Interpret these equations in economic terms.

\item Discuss how to calibrate the parameters of the model.

\item Write a DYNARE mod file for this model with a feasible calibration
  and compute the steady-state of the model either analytically or numerically.

\item Compare the steady-state values of an economy with a high proportion of Ricardian agents versus one with a very low proportion.
Try to interpret your results economically.
\end{enumerate}

\newpage

\section[Maximum likelihood estimation of An and Schorfheide (2007)]{Maximum likelihood estimation of An and Schorfheide (2007)\label{ex:AnSchorfheide2007ML}}
\paragraph{Description}
Consider the log-linearized version of the model of \textcite{An.Schorfheide_2007_BayesianAnalysisDSGE}
  which is a basic small-scale New-Keynesian model.

\paragraph{Data}
For the estimation, we will use quarterly U.S. data for quarter-to-quarter per capita GDP growth rates (YGR),
annualized quarter-to-quarter inflation rates (INFL), and annualized nominal interest rates (INT).
The three series are measured in percentages.

\paragraph{Parameters}
All of the 10 underlying model parameters as well as standard errors of the three structural shocks are to be estimated by maximizing the log-likelihood function with Dynare.

\paragraph{Codes}
A Dynare mod file is given in the file \texttt{as2007.mod}
  and \texttt{AnSchorfheide2007data.csv} contains the dataset.

\paragraph{Exercises}

\begin{enumerate}
\item What is the \emph{dilemma of absurd parameters} when doing a Maximum Likelihood estimation of DSGE models?

\item Add an \texttt{estimated\_params} block with feasible starting values, lower and upper bounds for all the parameters.

\item Estimate all the parameters with maximum likelihood.
Comment on the estimation results in terms of point estimates and standard errors.

\item Try to fix any estimation issues that you face
  by trying out different starting values, optimizers, or fixing some \emph{problematic} parameters.

\item Why is Maximum Likelihood estimation of DSGE models not straightforward?
What does a Bayesian approach offer to solve problems that arise in Maximum Likelihood estimation?

\end{enumerate}

\printbibliography%

\newpage

\section{Random-Walk-Metropolis-Hastings algorithm for AR{(1)}}
Re-consider the univariate first-order autoregressive process:
\begin{align*}
y_t = \rho y_{t-1} + u_t
\end{align*}
where we restrict \(0\leq \rho <1\) and \(u_t \sim \mathcal{N}(0,\sigma^2)\).
Simulated data for this process is given in the file \texttt{AR1Simulated.csv}.

\begin{enumerate}
\item Briefly outline the Random-Walk-Metropolis-Hastings algorithm.
What is the intuition behind this algorithm?

\item What are common choices to initialize the proposal distribution of the Random-Walk-Metropolis-Hastings algorithm?
Which one is preferred and why?

\item Our goal is to estimate the parameters \(\rho \) and \(\sigma \) using the Random-Walk-Metropolis-Hastings algorithm.
To this end:
\begin{itemize}
\item Assume that the prior distribution for
  \begin{itemize}
  \item \(\rho \) is Beta with both shape parameters being equal to 4: \texttt{betapdf(x,4,4)}.
  \item \(\sigma \) is Uniform on the interval \([0,2]\): \texttt{unifpdf(x,0,2)}.
  \end{itemize}
Plot the prior distribution for both parameters and comment whether this is a good choice.

\item Write a function \texttt{logposterior} that computes the log-posterior of the AR{(1)} process for any \(x=(\rho,\sigma)\) given the data.
Hints:
  \begin{itemize}
  \item We have already derived the log-likelihood function in the lecture:\\\texttt{logLikeARpNorm(x,data,1,0)}.
  \item Don't forget the log on the prior distributions.
  \item As a cross-check \texttt{logposterior{({[0.5,0.3]})}} \(\approx 42.1050\).
  \end{itemize}

\item Initialize the proposal distribution at the posterior mode and use the inverse hessian for the jumping covariance matrix.
Scale the jumping covariance by the factor \(c=2.5\).
Hints:
  \begin{itemize}
  \item You need to run a numerical optimizer on \texttt{-1*logposterior} to find the mode.
  \item The inverse hessian is given by \texttt{inv(hessian)} where \texttt{hessian} is either the output of the optimizer
  or you can use the auxiliary function \texttt{hessian\_numerical} to compute the hessian numerically.
  \end{itemize}
  If you fail this step, use something else as the jumping covariance matrix.

\item Initialize the chain \texttt{xpost} at the posterior mode and draw 100000 draws using the RW-MH algorithm.
  \begin{itemize}
  \item A new draw from the proposal is given by:
\begin{align*}
\text{\texttt{xtry = mvnrnd{(xpost{(:,i-1)},c*SigmaMH)}}}
\end{align*}
  \item Check whether \(\rho=xtry(1)\) and \(\sigma=xtry(2)\) are within the parameter space; if not penalize the log-posterior by -Inf.
  \item Do the Metropolis-Hastings accept-reject mechanism and store the accepted draws in \texttt{xpost}.
  \end{itemize}

\item Remove the first 50\% of the draws.
Plot the kernel density estimates of the marginal posterior distributions for \(\rho \) and \(\sigma \), e.g.\ \texttt{[frho, xrho] = ksdensity(rhopost);},
  and compare it to their prior distributions.

\end{itemize}

\end{enumerate}

\end{document}