% !TeX encoding = UTF-8
% !TeX spellcheck = en_US
% !TEX root = endterm_exam_WS2223.tex
\documentclass{article}
%%%%%%%%%%%%%%%%%%%%%%%%%%%%%%%%%%%%%%%%%%%%%%%%%%%%%%%%%%%%%%%%%%%%%%%%%%%%%%%%%%%%%%%%%%%%%%%%%%%%%%%%%%%%%%%%%%%%%%%%%%%%%%%%%%%%%%%%%%%%%%%%%%%%%%%%%%%%%%%%%%%%%%%%%%%%%%%%%%%%%%%%%%%%%%%%%%%%%%%%%%%%%%%%%%%%%%%%%%%%%%%%%%%%%%%%%%%%%%%%%%%%%%%%%%%%
\usepackage[a4paper,top=2cm]{geometry}
\usepackage{amssymb,amsmath,amsfonts}
\usepackage[english]{babel}
\usepackage[a4paper]{geometry}
\usepackage{enumitem}
\usepackage{booktabs}
\usepackage{csquotes}
\usepackage{graphicx}
\usepackage[numbered,framed]{matlab-prettifier}
\usepackage{url}
%\parindent0mm
%\parskip1.5ex plus0.5ex minus0.5ex
\usepackage[backend=biber,style=authoryear]{biblatex}
\addbibresource{../literature/_biblio.bib}
\begin{document}
	
\title{Quantitative Macroeconomics}
\author{Endterm Exam}
\date{Winter 2022/2023}
\maketitle

\section*{General information}
\begin{itemize}
	\item Answer \textbf{all} of the following \textbf{three} exercises in English.
	\item All assignments will be given the same weight in the final grade.
	\item Hand in your solutions before February, 23 2023 at 3pm.
	\item The solution files should contain your executable (and commented) script files
		as well as all additional documentation as \texttt{pdf}, not \texttt{txt}, \texttt{md}, \texttt{tex}, \texttt{doc} or \texttt{docx}.
	Your \texttt{pdf} files may also include scans or pictures of handwritten notes.
	\item Please e-mail ALL the solution files to \url{willi.mutschler@uni-tuebingen.de}.
	I will confirm the receipt of your work also by email (typically within the hour). If not, please resend it to me.
	\item All students must work on their own, please also give your student ID number and the name of the module you want to earn credits for.
	\item It is advised to regularly check Ilias and your emails in case of urgent updates.
	\item If there are any questions, do not hesitate to contact Willi Mutschler.
\end{itemize}

\newpage


\section{RBC model with leisure and non-Ricardian agents}
Consider the basic Real Business Cycle (RBC) model with leisure and non-Ricardian agents. Assume that there is a continuum of consumers given on the interval $[0,1]$. A proportion of the population, $\omega$, are Ricardian agents who have access to financial markets and are indexed by $i \in [0,\omega)$. The other part of the population, $1-\omega$, is composed of non-Ricardian agents who do not have access to financial markets and are indexed by $j \in (\omega,1]$.

A Ricardian household maximizes present as well as expected future utility
\begin{align*}
\underset{\{C_{i,t},I_{i,t},L_{i,t},K_{i,t}\}}{\max} E_t \sum_{s=0}^{\infty} \beta^{s} U_{i,t+s}
\end{align*}
with $\beta <1$ denoting the discount factor and $E_t$ is expectation given information at time $t$. The contemporaneous utility function 
\begin{align*}
U_{i,t} = \gamma \ln C_{i,t} + (1-\gamma) \ln{(1-L_{i,t})}
\end{align*}
has two arguments: consumption $C_{i,t}$ and labor $L_{i,t}$. The marginal utility of consumption is positive, whereas more labor reduces utility. Accordingly, $\gamma$ is the elasticity of substitution between consumption and labor. In each period the household takes the real wage $W_t$ as given and supplies perfectly elastic labor service to the representative firm. In return, she receives real labor income in the amount of $W_t L_{i,t}$ and, additionally, profits $\Pi_{i,t}$ from the firm as well as revenue from lending capital in the previous period $K_{i,t-1}$ at interest rate $R_t$ to the firms, as it is assumed that the firm and capital stock are owned by the Ricardian households. Income and wealth are used to finance consumption $C_{i,t}$ and investment $I_{i,t}$. In total, this defines the (real) budget constraint of the Ricardian agent:
\begin{align*}
C_{i,t} + I_{i,t} = W_t L_{i,t} + R_t K_{i,t-1} + \Pi_{i,t}
\end{align*}

The law of motion for capital $K_{i,t}$ at the end of period $t$ is given by
\begin{align*}
K_{i,t} = (1-\delta)K_{i,t-1} + I_{i,t}
\end{align*}
where $\delta$ is the depreciation rate. Assume that the transversality condition is full-filled.

A non-Ricardian household maximizes present as well as expected future utility
\begin{align*}
\underset{\{C_{j,t},L_{j,t}\}}{\max} E_t \sum_{s=0}^{\infty} \beta^{s} U_{j,t+s}
\end{align*}
The contemporaneous utility function is the same as for non-Ricardian households, i.e.
\begin{align*}
U_{j,t} = \gamma \ln C_{j,t} + (1-\gamma) \ln{(1-L_{j,t})}
\end{align*}
As non-Ricardian agents do not have access to the credit market, their (real) budget constraint is given by:
\begin{align*}
C_{j,t} = W_t L_{j,t}
\end{align*}

It is assumed that all agents, independently the group they belong to, are identical. Therefore, aggregate values (in per capita terms) are given by:
\begin{align*}
	C_t = \omega C_{i,t} + (1-\omega)C_{j,t}, && 		L_t = \omega L_{i,t} + (1-\omega)L_{j,t}, && K_t = \omega K_{i,t} &&	I_t = \omega I_{i,t}
\end{align*}
where the right two expressions are due to the fact that only Ricardian agents invest in physical capital.

Productivity $A_t$ is the driving force of the economy and evolves according to
\begin{align*}
\ln{A_{t}} &= \rho_A \ln{A_{t-1}}  + \varepsilon_t^A
\end{align*}
where $\rho_A$ denotes the persistence parameter and $\varepsilon_t^A$ is assumed to be normally distributed with mean zero and variance $\sigma_A^2$.

Real profits $\Pi_t = \omega \Pi_{i,t}$ of the representative firm are revenues from selling output $Y_t$ minus costs from labor $W_t L_t$ and renting capital $R_t K_{t-1}$:
\begin{align*}
\Pi_t = Y_{t} - W_{t} L_{t} - R_{t} K_{t-1}
\end{align*}	
The representative firm maximizes expected profits
\begin{align*}
\underset{\{L_{t},K_{t-1}\}}{\max} E_t \sum_{j=0}^{\infty} \beta^j Q_{t+j}\Pi_{t+j}
\end{align*}
subject to a Cobb-Douglas production function
\begin{align*}
Y_t = A_t K_{t-1}^\alpha L_t^{1-\alpha}
\end{align*}
The discount factor takes into account that firms are owned by the Ricardian households, i.e. $\beta^s Q_{t+s}$ is the present value of a unit of consumption in period $t+s$ or, respectively, the marginal utility of an additional unit of profit; therefore $Q_{t+s}=\frac{\partial U_{i,t+s}/\partial C_{i,t+s}}{\partial U_{i,t}/\partial C_{i,t}}$.

Finally, we have the non-negativity constraints	$K_{i,t} \geq0$, $C_{i,t} \geq 0$, $C_{j,t} \geq 0$, $0\leq L_{i,t} \leq 1$ and $0\leq L_{j,t} \leq 1$. Furthermore, clearing of the labor as well as goods market in equilibrium implies
\begin{align*}
Y_t = C_t + I_t
\end{align*}

\begin{enumerate}
	\item Briefly provide intuition behind the introduction of non-Ricardian households.
	\item Show that the first-order conditions of the agents are given by
		\begin{align*}
		E_t\left[\frac{C_{i,t+1}}{C_{i,t}}\right] = \beta E_t\left[1-\delta + R_{t+1}\right],&&		
		W_t = \frac{1-\gamma}{\gamma} \frac{C_{i,t}}{1-L_{i,t}},\\
		C_{j,t} = W_t L_{j,t}, &&
		W_t = \frac{1-\gamma}{\gamma} \frac{C_{j,t}}{1-L_{j,t}}.
		\end{align*}
	Interpret these equations in economic terms.
	
	\item Show that the first-order conditions of the representative firm are given by
	\begin{align*}
	W_t = (1-\alpha) A_t \left(\frac{K_{t-1}}{L_t}\right)^\alpha, &&	R_t = \alpha A_t \left(\frac{L_t}{K_{t-1}}\right)^{1-\alpha}
	\end{align*}
	Interpret these equations in economic terms.
	\item Discuss how to calibrate the parameters of the model.
	\item Write a DYNARE mod file for this model with a feasible calibration and compute the steady-state of the model either analytically or numerically.
	\item Compare the steady-state values of an economy with a high proportion of Ricardian agents versus one with a very low proportion.
	Try to interpret your results economically.
\end{enumerate} 

\newpage 

\section[Maximum likelihood estimation of Ireland (2004)]{Maximum likelihood estimation of Ireland (2004)\label{ex:Ireland2004ML}}
\paragraph{Description}
Consider a version of the model of \textcite{Ireland_2004_TechnologyShocksNew} who assesses which shocks are the major drivers of aggregate fluctuations.
To this end, he augments a basic small-scale New-Keynesian model with preference, cost-push, monetary and technology shocks.

\paragraph{Data}
For the estimation, we will use quarterly U.S. data from 1948:Q1 to 2003:Q1. Specifically:
\begin{itemize}
	\item Quarterly changes in seasonally adjusted real GDP figures, converted to per capita values by dividing by the civilian noninstitutional population aged 16 and over, are used to measure \textbf{output growth}.
	\item Quarterly changes in the seasonally adjusted GDP deflator provide the measure of \textbf{inflation}.
	\item Quarterly averages of daily values of the three-month U.S. Treasury bill rate provide the measure of the \textbf{nominal interest rate}.
\end{itemize}

\paragraph{Parameters}
Of the 14 underlying parameters in the model, two are held fix, $\beta=0.99$ and $\psi=0.1$, and are not estimated.
Hence, our interest centers around the other 12 parameters which we will estimate by maximizing the log-likelihood function.

\paragraph{Codes}
The model equations are given in the file \texttt{ireland\_2004\_preprocessing.m}, the steady-state is computed in the file \texttt{ireland\_2004\_steadystate.m},
  some incomplete information on parameters is provided in \texttt{ireland\_2004\_params.m}, and \texttt{ireland\_2004\_data.mat} contains the dataset.
The Maximum Likelihood estimation routines are given in the folder \texttt{MLRoutines} and some additional optimizers in the folder \texttt{OptimRoutines}.
A basic (incomplete) run script is provided in \texttt{XRunML\_ireland\_2004.m}.


\paragraph{Exercises}
\begin{enumerate}
	\item What is the \emph{dilemma of absurd parameters} when doing a Maximum Likelihood estimation of DSGE models?
	\item Add feasible starting values, lower and upper bounds to the file \texttt{ireland\_2004\_params.m}.
	\item Fix $\alpha_\pi$ to a very small number, e.g. 0.0001.
	Estimate the remaining 11 model parameters with maximum likelihood.
	Comment on the estimation results in terms of point estimates and standard errors.
	\item Now also include $\alpha_\pi$ to the set of estimated parameters.
	How does this affect the estimation results in terms of point estimates and standard errors?
	Try to explain this and point towards possible solutions.
\end{enumerate}

\newpage 

\section[Bayesian estimation of Smets and Wouters (2007)]{Bayesian estimation of Smets and Wouters (2007)\label{ex:SW2007RWMH}}
\paragraph{Description}
Consider the benchmark DSGE model of \textcite{Smets.Wouters_2007_ShocksFrictionsUS} which is fit to US macroeconomic data for the period 1966:1-2004:4.
This model constitutes one of the cornerstones that drive the development of medium-scale and large-scale DSGE models.
The model features monopolistic competition in the goods and labor markets and nominal frictions in the form of sticky prices and wages.
However, it allows both non-optimizing firms and households to index prices and wages, respectively, to a composite of steady-state and lagged inflation.
Households can also save in physical capital, with a one-period time to build before new investments turns into productive capital.
Moreover, the model features several real rigidities in the form of habit formation in consumption, investment adjustment costs, variable capital utilization, and fixed costs in production.
The linearized model consists of two blocks, one for the sticky prices and wages economy that determines the actual allocations
  and another one for the hypothetical allocation under flexible prices and wages in order to determine the potential allocation.
Both blocks are linked as monetary policy follows a Taylor rule with feedback not only to inflation deviations,
  but also to deviations from output and output-growth from its potential counterparts.
The model dynamics are driven by seven structural shocks.
Monetary policy shocks follow an AR{(1)} process and two additional inefficient markup shocks in wage and price setting follow an ARMA{(1,1)} process.
Four efficient shocks (total factor productivity, risk premium, investment-specific technology, and government spending shocks) follow AR{(1)} processes.

\paragraph{Data}
For estimation, we will consider data that comprises quarterly time series of the log difference of real GDP $d\log(GDP_t)$,
  log difference of real consumption $d\log(CONS_t)$, log difference of real investment $d\log(INV_t)$, log difference of real wage $d\log(WAGE_t)$,
  log hours worked $\log(HOURS_t)$, log difference of GDP deflator $d\log(GDPDEFLATOR_t)$ and the federal funds rate $FEDFUNDS_t$ for the US economy from 1966:Q1 to 2004:Q4.
The endogenous variables relate to the vector of observables through the relation
\begin{align}
	\begin{bmatrix} d\log(GDP_t)\\ d\log(CONS_t)\\ d\log(INV_t)\\ d\log(WAGE_t)\\ \log(HOURS_t)\\ d\log(GDPDEFLATOR_t)\\ FEDFUNDS_t\end{bmatrix}
	= \begin{bmatrix} \bar{\gamma}\\ \bar{\gamma}\\ \bar{\gamma}\\ \bar{\gamma}\\ \bar{l}\\ \bar{\pi}\\ \bar{r} \end{bmatrix}
	+ \begin{bmatrix} y_t - y_{t-1}\\c_t - c_{t-1}\\i_t - i_{t-1}\\w_t - w_{t-1}\\l_t\\\pi_t\\r_t\end{bmatrix} \label{eq:SW2007_measurement_equations}
\end{align}
where $d\log$ refers to the log difference of a variable.
$\bar{\gamma}$ is the quarterly trend growth rate of real GDP, consumption, investment and wages, 
  $\bar{l}$ is the quarterly steady-state number of hours worked,
  $\bar{\pi}$ is the quarterly steady-state inflation rate and 
  $\bar{r}$ is the steady-state nominal interest rate.
Note that we do not assume that the observed variables are measured with errors.

\paragraph{Parameters}
Of the 41 underlying parameters in the model, 5 are held fixed.
Our interest, hence, centers around the 36 dimensional posterior distribution resulting from the sampling density and prior distribution of the parameters. 

\paragraph{Codes}
An Dynare mod file is given in the (incomplete) file \texttt{sw2007\_.mod}
and \texttt{sw2007\_data\_v2.mat} contains the dataset.


\newpage
\paragraph{Exercises}
	\begin{enumerate}
	\item Add the measurement equations \eqref{eq:SW2007_measurement_equations} to your mod file.
	Make sure that the steady-state is correctly computed for these variables.
	Note that the underlying model variables are all expressed in log deviations from their corresponding steady-state,
	  e.g. $x_t = \log(X_t) - X$ where $X$ is the steady-state of $X_t$; hence the steady-state of $x_t$ is zero.
	\item Briefly outline the Random-Walk-Metropolis-Hastings algorithm.
	What is the intuition behind this algorithm?
	\item What are common choices to initialize the proposal distribution of the Random-Walk-Metropolis-Hastings algorithm?
	Which one is preferred and why?
	\item Pick three parameters of your choice and comment why the chosen prior distribution is a good or a bad one.
	\item Estimate the model parameters with the Random-Walk-Metropolis-Hastings algorithm:
	\begin{itemize}
		\item Initialize the proposal distribution at the posterior mode and use the inverse hessian for the jumping covariance matrix.
		\item Draw 250000 draws (or 100000 depending on your machine) using the RW-MH algorithm and keep 50\% of the last draws to conduct inference.\footnote{
		\textbf{Note that this could take up to 20 minutes. If it takes too much time, please decrease the number of draws to 100000.}}
		\item Tune the algorithm such that the acceptance rate is between 0.30 and 0.40.		
	\end{itemize}
	\item Compare the estimated marginal posterior with the marginal prior distribution.
	For which parameters is there learning from data, which parameters are mostly driven by their prior?
	\item What does it mean that \emph{a chain has converged} and why is this important?
	Comment on whether or not you think this is the case given your estimation results.	
\end{enumerate}

\newpage

\appendix

\printbibliography

\end{document}
