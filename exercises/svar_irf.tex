\section[Structural Impulse Response Function]{Structural Impulse Response Function\label{ex:StructuralImpulseResponseFunction}}
Consider the structural VAR{(p)} model
\begin{align*}
B_0y_t = B_1 y_{t-1} + \cdots  + B_{p} y_{t-p} + \varepsilon_{t}
\end{align*}
where the dimension of \(B_i\), \(i = 0,\ldots ,p\), is \(K \times K\).
The \(K \times 1\) vector of structural shocks \(\varepsilon_{t}\) is assumed to be white noise with covariance matrix \(E(\varepsilon_t \varepsilon_t') = I_K\).
That is, the elements of \(\varepsilon_t\) are mutually uncorrelated and also have a clear interpretation in terms on an underlying economic model.
The reduced-form VAR{(p)} model is given by
\begin{align*}
    y_t = \underbrace{B_0^{-1}B_1}_{A_1}y_t + \cdots  + \underbrace{B_0^{-1}B_p}_{A_p} y_{t-p} + \underbrace{B_0^{-1}\varepsilon_t}_{u_t}
\end{align*}
where the reduced-form error covariance matrix is \(E(u_t u_t')=\Sigma_u = B_0^{-1}B_0^{-1'}\).
Going back and forth between the structural and the reduced-form representation requires knowledge of the structural impact matrix \(B_0^{-1}\).
For now, assume that \(B_0^{-1}\) is a known matrix.
We are interested in the response of each element of \(y_t\) to a one-time impulse in \(\varepsilon_{t}\):
\begin{align*}
    \frac{\partial y_{t+h}}{\partial \varepsilon_{t}'} = \Theta_h, \quad h=0,1,2,\ldots ,H
\end{align*}
where \(\Theta_h\) is a \(K\times K\) matrix with individual elements \(\theta_{jk,h}=\frac{\partial y_{j,t+h}}{\partial \varepsilon_{k,t}}\).

\begin{enumerate}
\item Usually the objective is to plot the responses of each variable to each structural shock.
How many so-called impulse response functions do we need to plot?

\item Consider the VAR{(1)} representation of the VAR{(p)} process, i.e.\
\begin{align*}
Y_t = A Y_{t-1} + U_t
\end{align*}
where
\begin{footnotesize}
\begin{align*}
Y_t = \begin{pmatrix}
    y_t\\ \vdots\\ y_{t-p+1}
    \end{pmatrix}, 
    A = \begin{pmatrix}
    A_1 & A_2 & \cdots  & A_{p-1} & A_p\\
    I_k &   0 & \cdots  & 0       & 0\\
    0   &  I_K& \cdots  & 0       & 0\\
    \vdots & \vdots & \ddots &\vdots & \vdots\\ 
    0& 0 & \cdots  &I_k & 0
    \end{pmatrix},
    U_t = \begin{pmatrix} u_t\\0\\ \vdots \\0 \end{pmatrix}
\end{align*}
\end{footnotesize}
Show that
\begin{align*}
    y_{t+h} = J A^{h+1} Y_{t-1}+ \sum_{j=0}^h J A^j J' u_{t+h-j}
\end{align*}
where \(J=[I_K, 0_{K\times K(p-1)}]\).

\item Derive an expression for \(\Theta_h\) in terms of \(J\), \(A\) and \(B_0^{-1}\).

\item How would you infer from the response of the inflation rate, say \(\Delta p_t\), the implied response of the log price level \(p_t\)?
 	
\item Write a function that plots the IRFs given inputs \(A\), \(B_0^{-1}\), number of lags \(p\), maximum horizon \(H\),
  an indicator vector whether the cumulative sum of the IRFs should be computed, and string arrays for the names of the variables and shocks.
\end{enumerate}

\paragraph{Readings}
\begin{itemize}
	\item \textcite[Ch.~4.1]{Kilian.Lutkepohl_2017_StructuralVectorAutoregressive}
\end{itemize}

\begin{solution}\textbf{Solution to \nameref{ex:StructuralImpulseResponseFunction}}
\ifDisplaySolutions
\input{exercises/svar_irf_solution.tex}
\fi
\newpage
\end{solution}