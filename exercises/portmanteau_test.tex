\section[Portmanteau Test For Residual Autocorrelation]{Portmanteau Test For Residual Autocorrelation\label{ex:PortmanteauTestResidualAutocorrelation}}
The portmanteau test checks the null hypothesis
  that there is no remaining residual autocorrelation at lags \(1\) to \(h\)
  against the alternative that at least one of the autocorrelations is nonzero.
In other words, the pair of hypotheses:
\begin{align*}
H_0:\rho_u(1)=\rho_u(2)=\cdots =\rho_u(h) = 0
\end{align*}
versus:
\begin{align*}
H_1: \rho_u(j) \neq 0 \text{~for at least one~} j=1,\ldots ,h
\end{align*}
where \(\rho_u(j) = Corr(u_t, u_{t-j})\) denotes an autocorrelation coefficient of the residual series.

Consider the Box-Pierce test statistic \(Q_h\)
\begin{align*}
Q_h = T \sum_{j=1}^h \hat{\rho}^2_u(j)
\end{align*}
which has an approximate \(\chi^2(h-p)\)-distribution if the null hypothesis holds
  and \(T\) is the length of the residual series.
The null hypothesis of no residual autocorrelation is rejected for large values of the test statistic.

\begin{itemize}

\item
Load quarterly data for the price index of US Gross National Product given in \texttt{gnpdeflator.csv}.
This is a chain-type price index with basis year 2005.
The data is seasonally adjusted and spans the period from 1954.Q4 to 2007.Q4.

\item
Compute the inflation series.
That is, take the first difference of the log of gnpdeflator.

\item
Use the Akaike information criteria to determine the lag length \(\hat{p}\).

\item
Estimate two models with {OLS}:
  (i) an AR{(\(\hat{p}\))} model and
  (ii) an AR{(1)} model.

\item
Set \(h=\hat{p}+10\) and compute \(Q_h\) as well as the corresponding p-value for both models.

\item
Comment, based on your findings, whether the residuals are white noise.
\end{itemize}

\paragraph{Readings:}
\begin{itemize}
\item \textcite{Lutkepohl_2004_UnivariateTimeSeries}
\end{itemize}

\begin{solution}\textbf{Solution to \nameref{ex:PortmanteauTestResidualAutocorrelation}}
\ifDisplaySolutions%
\lstinputlisting[style=Matlab-editor,basicstyle=\mlttfamily,title=\lstname]{progs/matlab/portmanteauTest.m}
The Null hypothesis of \textbf{no remaining residual autocorrelation} can be rejected for the AR{(1)} model,
  but not for the AR{(\(\hat{p}\))} model.
This implies that the residuals in the AR{(1)} model are NOT white noise,
  while the residuals in the AR{(\(\hat{p}\))} model are LIKELY to be white noise.
In sum, both the information criteria as well as the Portmanteau test favor the AR{(\(\hat{p}\))} model over the AR{(1)} model.
\fi
\newpage
\end{solution}