\section[How Well Does the IS-LM Model Fit Postwar US Data]{How Well Does the IS-LM Model Fit Postwar US Data?\label{ex:svarISLM}}
Consider a quarterly model for \(y_t = (\Delta gnp_t, \Delta i_t, i_t-\Delta p_t, \Delta m_t - \Delta p_t)'\),
  where \(gnp_t\) denotes the log of GNP, \(i_t\) the nominal yield on three-month Treasury Bills,
  \(\Delta m_t\) the growth in M1 and \(\Delta p_t\) the inflation rate in the CPI\@.
There are four shocks in the system: an aggregate supply (AS), a money supply (MS), a money demand (MD) and an aggregate demand (IS) shock.
Ignoring the lagged dependent variables for \textbf{expository} purposes (\(B_1=\cdots =B_p=0\)),
  the unrestricted structural VAR model can be simply written as \(B_{0} y_t = \varepsilon_t\).
That is:
\begin{align}
\Delta gnp_t &= -b_{12}\Delta i_t -b_{13}(i_t-\Delta p_t) -b_{14}(\Delta m_t-\Delta p_t) + \varepsilon_t^{AS} \label{eq:AS}\\
\Delta i_t &= -b_{21}\Delta gnp_t -b_{23}(i_t-\Delta p_t) -b_{24}(\Delta m_t-\Delta p_t) + \varepsilon_t^{MS} \label{eq:MS}\\
i_t - \Delta p_t &= -b_{31}\Delta gnp_t -b_{32}\Delta i_t -b_{34}(\Delta m_t-\Delta p_t) + \varepsilon_t^{MD} \label{eq:MD}\\
\Delta m_t - \Delta p_t &= -b_{41}\Delta gnp_t -b_{42}\Delta i_t - b_{43} (i_t-\Delta p_t) + \varepsilon_t^{IS} \label{eq:IS}
\end{align}
where \(b_{ij}\) denotes the \(ij\)th element of \(B_{0}\).
Consider the following identification restrictions:

\begin{itemize}
\item Money supply shocks do not have contemporaneous effects on output growth, i.e.\
\begin{align*}
\frac{\partial \Delta gnp_t}{\partial \varepsilon_t^{MS}}=0
\end{align*}

\item Money demand shocks do not have contemporaneous effects on output growth, i.e.\
\begin{align*}
\frac{\partial \Delta gnp_t}{\partial \varepsilon_t^{MD}}=0
\end{align*}

\item Monetary authority does not react contemporaneously to changes in the price level.\\Hint: compute from equation~\eqref{eq:MS}:
\begin{align*}
\frac{\partial \Delta i_t}{\partial \Delta p_t}=0
\end{align*}

\item Money supply shocks, money demand shocks and aggregate demand shocks do not have long-run effects on the log of real {GNP}:
\begin{align*}
\frac{\partial gnp_t}{\partial \varepsilon_t^{MS}}=0,\qquad \frac{\partial gnp_t}{\partial \varepsilon_t^{MD}}=0,\qquad \frac{\partial gnp_t}{\partial \varepsilon_t^{IS}}=0
\end{align*}

\item The structural shocks are uncorrelated with covariance matrix \(E(\varepsilon_t \varepsilon_t')=\Sigma_{\varepsilon} \).
In other words, the variances are \textbf{not} normalized.
\end{itemize}
Solve the following exercises:

\begin{enumerate}

\item Derive the implied exclusion restrictions on the matrices \(B_{0}\), \(B_{0}^{-1}\) and \(\Theta(1)\).

\item Consider data given in the csv file \texttt{gali1992.csv}.
Estimate a VAR{(4)} model with a constant.

\item Estimate the structural impact matrix using a nonlinear equation solver,
  i.e.\ the objective is to find the unknown elements of \(B_{0}^{-1}\) and the diagonal elements of \(\Sigma_{\varepsilon} \) such that
\begin{align*}
\begin{bmatrix}
vech(B_{0}^{-1} \Sigma_{\varepsilon} B_{0}^{-1'}-\hat{\Sigma}_u)\\
\text{short-run restrictions on~}B_{0} \text{~and~} B_{0}^{-1} \\
\text{long-run restrictions on~}\Theta(1)\\
\end{bmatrix}
\end{align*}
is minimized.
Normalize the shocks such that the diagonal elements of \(B_{0}^{-1}\) are positive.

\item Use the implied estimates of \(B_{0}^{-1}\) and \(\Sigma_{\varepsilon}\) to plot the structural impulse responses functions
  for (i) real GNP, (ii) the yield on Treasury Bills, (iii) the real interest rate and (iv) real money growth.
Add 68\% and 95\% confidence intervals using a bootstrap approach.
\end{enumerate}

\paragraph{Readings}
\begin{itemize}
\item \textcite{Gali_1992_HowWellDoes}
\end{itemize}

\begin{solution}\textbf{Solution to \nameref{ex:svarISLM}}
\ifDisplaySolutions%
\begin{enumerate}

\item First, let's rewrite the equations in matrix form:
\begin{align*}
\underbrace{%
\begin{bmatrix}
1 & b_{12} & b_{13} & b_{14}\\
b_{21} & 1 & b_{23} & b_{24}\\
b_{31} & b_{32} & 1 & b_{34}\\
b_{41} & b_{42} & b_{43} & 1
\end{bmatrix}}_{B_{0}}
\begin{bmatrix}
\Delta gnp_t\\
\Delta i_t\\
i_t - \Delta p_t\\
\Delta m_t - \Delta p_t
\end{bmatrix}
&=
\begin{bmatrix}
\varepsilon_t^{AS}\\
\varepsilon_t^{MS}\\
\varepsilon_t^{MD}\\
\varepsilon_t^{IS}
\end{bmatrix}
\\
\begin{bmatrix}
\Delta gnp_t\\
\Delta i_t\\
i_t - \Delta p_t\\
\Delta m_t - \Delta p_t
\end{bmatrix}
&=
\underbrace{%
\begin{bmatrix}
b^*_{11} & b^*_{12} & b^*_{13} & b^*_{14}\\
b^*_{21} & b^*_{22} & b^*_{23} & b^*_{24}\\
b^*_{31} & b^*_{32} & b^*_{33} & b^*_{34}\\
b^*_{41} & b^*_{42} & b^*_{43} & b^*_{44}
\end{bmatrix}}_{B^{-1}_0}
\begin{bmatrix}
\varepsilon_t^{AS}\\
\varepsilon_t^{MS}\\
\varepsilon_t^{MD}\\
\varepsilon_t^{IS}
\end{bmatrix}
\end{align*}
The long-run multiplier matrix is given by:
\begin{align*}
\Theta(1) = {A(1)}^{-1} B^{-1}_0
\end{align*}

Now let's derive the restrictions on the impact matrix \(B_{0}^{-1}\):
\begin{itemize}
\item Money supply shocks do not have contemporaneous effects on output growth, i.e.\
\begin{align*}
\frac{\partial \Delta gnp_t}{\partial \varepsilon_t^{MS}}= b^*_{12} = 0
\end{align*}

\item Money demand shocks do not have contemporaneous effects on output growth, i.e.\
\begin{align*}
\frac{\partial \Delta gnp_t}{\partial \varepsilon_t^{MD}}= b^*_{13} = 0	
\end{align*}
\end{itemize}
Summarizing this yields two restrictions:
\begin{align*}
B_{0}^{-1} =
\begin{pmatrix}
* & 0 & 0 & *\\
* & * & * & *\\
* & * & * & *\\
* & * & * & *
\end{pmatrix}
\end{align*}

Next, let's derive the restrictions on the structural matrix \(B_{0}\):
\begin{itemize}
\item Monetary authority does not react contemporaneously to changes in the price level.
This can be computed directly from equation~\eqref{eq:MS}:
\begin{align*}
\frac{\partial \Delta i_t}{\partial \Delta p_t}= b_{23} + b_{24} = 0 \Leftrightarrow b_{23} = -b_{24}
\end{align*}
\end{itemize}
Summarizing this yields one restriction:
\begin{align*}
B_{0} =
\begin{pmatrix}
1 & * & * & *\\
* & 1 & -b_{24} & b_{24}\\
* & * & 1 & *\\
* & * & * & 1
\end{pmatrix}
\end{align*}

\begin{itemize}
\item Money supply shocks, money demand shocks and aggregate demand shocks do not have long-run effects on the log of real GNP.
\end{itemize}
The restrictions on the long-run multiplier matrix are thus:
\begin{align*}
\Theta(1) =
\begin{pmatrix}
* & 0 & 0 & 0\\
* & * & * & *\\
* & * & * & *\\
* & * & * & *
\end{pmatrix}
\end{align*}
This yields three restrictions.

In total we have \(2+1+3=6\) restrictions, which is equal to the required number of \(K(K-1)/2=6\) of an exactly identified SVAR model.

Lastly, we need to keep in mind that the variance of the structural shocks is not normalized:
\begin{align*}
\Sigma_{\varepsilon} =
\begin{pmatrix}
\sigma_{11} & 0 & 0 & 0\\
0 & \sigma_{22} & 0 & 0\\
0 & 0 & \sigma_{33} & 0\\
0 & 0 & 0 & \sigma_{44}
\end{pmatrix}
\end{align*}

\item[2/3/4]
Here is the helper function to impose the restrictions:
\lstinputlisting[style=Matlab-editor,basicstyle=\mlttfamily,title=\lstname]{progs/matlab/gali1992_f.m}
The main code might look like this:
\lstinputlisting[style=Matlab-editor,basicstyle=\mlttfamily,title=\lstname]{progs/matlab/gali1992.m}

\end{enumerate}
\fi
\newpage
\end{solution}