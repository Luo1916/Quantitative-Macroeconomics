\begin{enumerate}
\item \lstinputlisting[style=Matlab-editor,basicstyle=\mlttfamily,title=\lstname]{progs/matlab/matrixAlgebraEigenvalues.m}
In the univariate AR{(1)} model we would check whether the autocorrelation coefficient is between -1 and 1, i.e.\
  whether \(|\phi| = |A|<1\) such that \(\sum_{j=0}^\infty {(AL)}^j=1/(1-AL)\)
  where \(L\) is the lag operator.
In the multivariate case, we want to check the same thing, i.e.\
  \(\sum_{j=0}^\infty {(AL)}^j={(1-AL)}^{-1}\).
Note that \(A\) is a square matrix and taking the power of a matrix is not a trivial task.
One convenient way to do so, is to consider an eigenvalue decomposition (if it exists):
\[A= Q \Lambda Q^{-1}\]
  where \(Q\) is a square matrix whose columns contain the eigenvectors \(q_i\)
  corresponding to the eigenvalues \(\lambda_i\) found on the diagonal of \(\Lambda = {[\lambda_i]}_{ii}\).
Moreover, \(\Lambda \) is a diagonal matrix and \(Q\) is an orthogonal matrix \(Q^{-1}=Q'\).
Using this decomposition one can show that it is very easy to compute any power of a matrix:
\begin{itemize}
\item matrix inverse \(A^{-1} = Q \Lambda^{-1} Q^{-1}\) where the inverse of \({[\Lambda^{-1}]}_{ii} = 1/\lambda_i\) is very easy to calculate as it is a diagonal matrix
\item matrix powers: \(A^2=(Q\Lambda Q^{-1})(Q\Lambda Q^{-1})=Q\Lambda (Q^{-1}Q) \Lambda Q^{-1} = Q \Lambda^2 Q^{-1}\) or more generally: \(A^j = Q \Lambda^j Q^{-1}\).
\end{itemize}
So, for \(\sum_{j=0}^\infty {(AL)}^j={(1-AL)}^{-1}\) we need that \(\lim_{j\rightarrow \infty}\Lambda^j=0\).
As this is a diagonal matrix, the task simplifies as we only need to look at each eigenvalue whether it is between -1 and 1: \(|\lambda_i|<1\).
In other words, for VAR{(1)} systems \(y_t = c + A y_{t-1} + u_t\) we need to check whether the eigenvalues of \(A\) are inside the unit circle.
If they are, then the VAR{(1)} model is said to be both stable and covariance-stationary.

\item Example for vectorization and Kronecker product:
\begin{align*}
\underbrace{vec
  \begin{pmatrix}
  1&3&2\\
  1&0&0\\
  1&2&2
  \end{pmatrix}
}_{3\times3}
=
\underbrace{%
  \begin{pmatrix}1 \\1 \\1 \\3 \\0 \\2 \\2 \\ 0 \\2\end{pmatrix}
}_{9\times1}
,\qquad 
\underbrace{%
  \begin{pmatrix}
  1&3&2\\
  1&0&0\\
  1&2&2
  \end{pmatrix}
}_{3\times3}
\otimes
\underbrace{%
  \begin{pmatrix}
  0&5\\
  5&0\\
  1&1
  \end{pmatrix}
}_{3\times2}
=
\underbrace{%
  \begin{pmatrix}
  1 \cdot \begin{pmatrix} 0 & 5 \\ 5 & 0 \\ 1 & 1 \end{pmatrix} &
  3 \cdot \begin{pmatrix} 0 & 5 \\ 5 & 0 \\ 1 & 1 \end{pmatrix} &
  2 \cdot \begin{pmatrix} 0 & 5 \\ 5 & 0 \\ 1 & 1 \end{pmatrix}
  \\
  1 \cdot \begin{pmatrix} 0 & 5 \\ 5 & 0 \\ 1 & 1 \end{pmatrix} &
  0 \cdot \begin{pmatrix} 0 & 5 \\ 5 & 0 \\ 1 & 1 \end{pmatrix} &
  0 \cdot \begin{pmatrix} 0 & 5 \\ 5 & 0 \\ 1 & 1 \end{pmatrix}
  \\
  1 \cdot \begin{pmatrix} 0 & 5 \\ 5 & 0 \\ 1 & 1 \end{pmatrix}&
  2 \cdot \begin{pmatrix} 0 & 5 \\ 5 & 0 \\ 1 & 1 \end{pmatrix}&
  2 \cdot \begin{pmatrix} 0 & 5 \\ 5 & 0 \\ 1 & 1 \end{pmatrix}
  \end{pmatrix}
}_{9\times6}
\end{align*}
Using this definition, we can show that \(vec(DEF) = (F' \otimes D) vec(E)\) using e.g.\ a symbolic toolbox:
\lstinputlisting[style=Matlab-editor,basicstyle=\mlttfamily,title=\lstname]{progs/matlab/matrixAlgebraKroneckerFormula.m}
Of course you can do this on paper as well:

\begingroup\scriptsize
\begin{align*}
DEF &= D \begin{pmatrix} e_1 & e_2 & \cdots & e_p \end{pmatrix}
\begin{pmatrix}
f_{11} &f_{12} &\cdots &f_{1k} \\
f_{21} &f_{22} &\cdots &f_{2k} \\
\vdots &\vdots &\vdots &\vdots \\
f_{p1} &f_{p2} &\cdots &f_{pk}
\end{pmatrix}
\\
&= D \underbrace{%
  \begin{pmatrix}
  e_1 f_{11} + e_2 f_{21} + \cdots + e_p f_{p1}, &
  e_1 f_{12} + e_2 f_{22} + \cdots + e_p f_{p2}, &
  \dots ,&
  e_1 f_{1k} + e_2 f_{2k} + \cdots + e_p f_{pk}
  \end{pmatrix}
}_{n\times k}
\end{align*}
\endgroup
Vectorizing:
\begin{align*}
vec(DEF) &=
\begin{pmatrix}
f_{11} D e_1 + f_{21} D e_2 + \cdots + f_{p1} D e_p \\
f_{12} D e_1 + f_{22} D e_2 + \cdots + f_{p2} D e_p \\
\vdots \\
f_{1k} D e_1 + f_{2k} D e_2 + \cdots + f_{pk} D e_p
\end{pmatrix}
=
\begin{pmatrix}
f_{11} D & f_{21} D & \cdots & f_{p1} D \\
f_{12} D & f_{22} D & \cdots & f_{p2} D \\
\vdots & \vdots & \vdots & \vdots \\
f_{1k}D & f_{2k} D & \cdots & f_{pk} D
\end{pmatrix}
\begin{pmatrix} e_1 \\ e_2 \\ \vdots \\ e_p \end{pmatrix}
\\
&= \left(F'\otimes D\right) vec(E)
\end{align*}

\item An orthogonal matrix is characterized by \(R'=R^{-1}\) and therefore \(R'R=R R' = I\).
Here:
\begin{align*}
R'R =
\begin{pmatrix}
{(\cos(\phi))}^2 + {(\sin(\phi))}^2 & -\cos(\phi)\sin(\phi) + \sin(\phi)\cos(\phi)
\\
-\sin(\phi)\cos(\phi) + \cos(\phi)\sin(\phi) & {(\sin(\phi))}^2 + {(\cos(\phi))}^2
\end{pmatrix}
\end{align*}
with \({(\cos(\phi))}^2 + {(\sin(\phi))}^2 = 1\) (so-called trigonometric Pythagoras).

\lstinputlisting[style=Matlab-editor,basicstyle=\mlttfamily,title=\lstname]{progs/matlab/matrixAlgebraRotation.m}

\(R\) is called a rotation matrix,
  because it rotates vectors or objects in the Euclidian space without stretching or shrinking the object.
\begin{center}
\includegraphics[width=.5\textwidth]{plots/Rotation.png}
\end{center}
In this example the matrix R rotates the vector counter-clockwise given angle \(\phi \).
An active rotation means that the vector is multiplied by the rotation matrix
  and this rotates the vector counterclockwise \(x' = Rx\).
A passive rotation means that the coordinate system is rotated and therefore the vector is also rotated: \(x' = R^{-1} x\).
Later on we will need rotation matrices for identification of structural shocks!

\item \lstinputlisting[style=Matlab-editor,basicstyle=\mlttfamily,title=\lstname]{progs/matlab/matrixAlgebraCholesky.m}
\begin{align*}
\underbrace{%
  \begin{pmatrix}
  1 & 0   & 0 \\
  0 & 1   & 0 \\
  0 & 0.5 & 1
  \end{pmatrix}
}_W
\underbrace{%
  \begin{pmatrix}
  2.25 & 0 & 0 \\
  0    & 1 & 0 \\
  0    & 0 & 0.49
  \end{pmatrix}
}_{\Sigma_\varepsilon}
\underbrace{%
  \begin{pmatrix}
  1 & 0 & 0   \\
  0 & 1 & 0.5 \\
  0 & 0 & 1
  \end{pmatrix}
}_{W'}
=
\underbrace{%
  \begin{pmatrix}
  2.25 & 0  &  0   \\
  0    & 1   & 0.5 \\
  0    & 0.5 & 0.74
  \end{pmatrix}
}_{\Sigma}
\end{align*}

\item Solving this equation can be done either analytically or using an algorithm:

\begin{enumerate}

\item Analytically:
\begin{align*}
vec(\Sigma_y) &= vec(A \Sigma_y A') + vec(\Sigma_u) = (A \otimes A)vec(\Sigma_y) + vec(\Sigma_u)
\\
(I-A\otimes A)vec(\Sigma_y) &= vec(\Sigma_u)
\\
vec(\Sigma_y) &= {(I-A\otimes A)}^{-1}vec(\Sigma_u)
\end{align*}

\item Doubling algorithm:
\lstinputlisting[style=Matlab-editor,basicstyle=\mlttfamily,title=\lstname]{progs/matlab/dlyapdoubling.m}
The basic idea of the doubling algorithm is to start at some \(\Sigma_{y,0}\)
  and find new values for \(\Sigma_{y,i+1}\) using the equation \(A \Sigma_{y,i} A' + \Sigma_u\)
  until the difference \(\Sigma_{y,i+1} - \Sigma_{y,i}\) becomes very small
  or a certain maximum number of iterations is reached.

The doubling algorithm, however, allows one to pass in one iteration
  from \(\Sigma_{y,i}\) to \(\Sigma_{y,2i}\) rather than \(\Sigma_{y,i+1}\),
  provided that one updates three other matrices.
There are also other (generalized) algorithms to solve such matrix Lyapunov (or Sylvester) equations.

\item Comparison:
\lstinputlisting[style=Matlab-editor,basicstyle=\mlttfamily,title=\lstname]{progs/matlab/matrixAlgebraLyapunovComparison.m}
The doubling algorithm is faster than the analytical closed-form expression based on the Kronecker product.
\end{enumerate}

\end{enumerate}