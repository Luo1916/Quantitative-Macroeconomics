\section[Law Of Large Numbers]{Law Of Large Numbers\label{ex:LawOfLargeNumbers}}
Let \(Y_{1},Y_{2},\ldots \) be an iid sequence of arbitrarily distributed random variables
  with finite variance \(\sigma_Y ^{2}\) and expectation \(\mu \).
Define the sequence of random variables
\begin{equation*}
\overline{Y}_{T}=\frac{1}{T}\sum_{t=1}^{T}Y_{t}
\end{equation*}
\begin{enumerate}

\item
Briefly outline the intuition behind the \enquote{law of large numbers}.
What are the differences between \enquote{almost-sure convergence}
  and \enquote{convergence in probability}?

\item Write a program to illustrate the law of large numbers for uniformly distributed random variables
  (you may also try different distributions such as normal, gamma, geometric, student's t with finite or infinite variance).
To this end, do the following:
\begin{itemize}
  \item
  Set \(T=10000\) and initialize the \(T \times 1\) output vector \(u\).

  \item
  Choose values for the parameters of the uniform distribution.
  Note that \(E[u] = (a+b)/2\), where \(a\) is the lower and \(b\) the upper bound of the uniform distribution.

  \item
  For \(t=1,\ldots,T\) do the following computations:
  \begin{itemize}
    \item
    Draw \(t\) random variables from the uniform distribution with lower bound \(a\) and upper bound \(b\).
    \item
    Compute and store the mean of the drawn values in your output vector at position \(t\).
  \end{itemize}

  \item
  Plot your output vector and add a line to indicate the theoretical mean of the uniform distribution.
\end{itemize}

\item
Now suppose that the sequence \(Y_{1},Y_{2},\ldots \) is an AR{(1)} process:
\[Y_{t} =\phi Y_{t-1} +\varepsilon _{t}\]
where \(\varepsilon _{t}\sim iid(0,\sigma _{\varepsilon }^{2})\) is not necessarily normally distributed and \(|\phi |<1\).
Illustrate numerically that the law of large numbers still holds despite the intertemporal dependence.
\end{enumerate}

\paragraph{Readings}
\begin{itemize}
\item \textcite[App. C]{Lutkepohl_2005_NewIntroductionMultiple}
\item \textcite[App. C]{Neusser_2016_TimeSeriesEconometrics}
\item \textcite{Ploberger_2010_LawLargeNumbers}
\item \textcite[Ch. 3]{White_2001_AsymptoticTheoryEconometricians}
\end{itemize}

\begin{solution}\textbf{Solution to \nameref{ex:LawOfLargeNumbers}}
\ifDisplaySolutions%
\input{exercises/law_of_large_numbers_solution.tex}
\fi
\newpage
\end{solution}