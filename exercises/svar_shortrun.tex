\section[Non-recursively Identified Models By Short-Run Restrictions]{Non-recursively Identified Models By Short-Run Restrictions\label{ex:NonRecursivelyIdentifiedModelsShortRun}}
Consider a quarterly model of US monetary policy.
Let \(y_t=(\Delta p_t, \Delta gnp_t, i_t, \Delta m_t)'\),
  where \(p_t\) refers to the log of the GNP deflator,
  \(gnp_t\) to the log of real GNP,
  \(i_t\) to the federal funds rate, averaged by quarter,
  and \(m_t\) to the log of money aggregate M1.
The data is given in \texttt{Keating1992.csv}.

The structural shock vector \(\varepsilon_t = (\varepsilon_t^{AS},\varepsilon_t^{IS},\varepsilon_t^{MS},\varepsilon_t^{MD})\)
  includes an aggregate supply shock, an IS shock, a money supply shock, and a money demand shock.
The structural model can be written as:
\begin{align*}
\begin{pmatrix}
u_t^p\\u_t^{gnp}\\u_t^{i}\\u_t^m
\end{pmatrix} = 
\begin{pmatrix}
\varepsilon_{t}^{AS}\\
-b_{21,0}u_t^p - b_{23,0} u_t^i- b_{24,0}u_t^m +\varepsilon_{t}^{IS}\\
-b_{34,0}u_t^m + \varepsilon_{t}^{MS}\\
-b_{41,0}(u_t^{gnp}+u_t^p)-b_{43,0}u_t^i +\varepsilon_{t}^{MD}
\end{pmatrix}
\end{align*}
Furthermore, it is assumed that \(E(\varepsilon_t \varepsilon_t')=\Sigma_{\varepsilon}\) is NOT the identity matrix;
  hence, we will use the normalization rule that the diagonal elements of \(B_{0}\) equal unity.

\begin{enumerate}
\item Estimate the reduced-form covariance matrix of a VAR model with four lags and an intercept with ordinary least squares.

\item Provide intuition behind the structural model and derive the \(B_{0}\) matrix.
Note that we are imposing restrictions on both \(B_{0}\) and \(\Sigma_{\varepsilon} \), but not on \(B_{0}^{-1}\).

\item Estimate \(B_{0}\) and \(\Sigma_{\varepsilon}\) using the nonlinear equation solver \texttt{fsolve}.
To this end, first set up a function that computes
\begin{align*}
F_{SR}([B_{0}, diag(\Sigma_{\varepsilon})]) = \begin{bmatrix} vech\left(B_{0}^{-1} \Sigma_{\varepsilon} B_{0}^{-1'} - \hat{\Sigma}_u\right)\\\text{restrictions on }B_{0}\end{bmatrix}=0
\end{align*}
where \(diag(\Sigma_{\varepsilon})\) denotes only the diagonal elements of \(\Sigma_{\varepsilon}\).
Note that these diagonal elements as well as \(B_{0}\) are stacked into a matrix or vector which is then used as the input for the numerical optimizer.
Use feasible options for the optimizer (e.g.\ as in the previous exercises).

\item Use the estimated structural impact matrix \(B_{0}^{-1}\Sigma_{\varepsilon}^{1/2}\) to plot the structural impulse response functions using \texttt{irfPlots.m}.
Interpret the effects of an aggregate supply shock on prices, real GNP, the Federal Funds rate and M1.
\end{enumerate}

\paragraph{Readings}
\begin{itemize}
\item \textcite[Ch.~8-9]{Kilian.Lutkepohl_2017_StructuralVectorAutoregressive}
\end{itemize}

\begin{solution}\textbf{Solution to \nameref{ex:NonRecursivelyIdentifiedModelsShortRun}}
\ifDisplaySolutions%
\begin{enumerate}
\item The OLS estimate of the reduced-form covariance matrix \(\Sigma_{u}\) is
\begin{align*}
\widehat{\Sigma}_u =
\begin{pmatrix}
 0.061054  & -0.015282 &  0.042398 &  0.0037836 \\
-0.015282  &  0.52305  &  0.07967  &  0.030602 \\
 0.042398  &  0.07967  &  0.71689  & -0.2451 \\
 0.0037836 &  0.030602 & -0.2451   &  1.1093 \\
\end{pmatrix} 
\end{align*}

\item In order to identify the structural shocks from \(\Sigma_{u} = B_{0}^{-1} \Sigma_{\varepsilon} B_{0}^{-1'}\),
  we require at least \(4(4-1)/2=6\) additional restrictions on \(B_{0}\) or \(B_{0}^{-1}\).
In this exercise, we'll do this on \(B_{0}\) and not its inverse; that is, we can rewrite the equations in matrix notation:
\begin{align*}
\underbrace{%
  \begin{pmatrix}
  1 & 0 & 0 & 0 \\
  b_{21,0} & 1 & b_{23,0} & b_{24,0} \\
  0 & 0 & 1 & b_{34,0} \\
  b_{41,0} & b_{41,0} & b_{43,0} & 1
  \end{pmatrix}
}_{B_{0}}
\underbrace{%
  \begin{pmatrix}
  u_t^p \\
  u_t^{gnp} \\
  u_t^{i} \\
  u_t^{m}
  \end{pmatrix}
}_{u_t}
=
\underbrace{%
  \begin{pmatrix}
  \varepsilon_t^{AS} \\
  \varepsilon_t^{IS} \\
  \varepsilon_t^{MS} \\
  \varepsilon_t^{MD}
  \end{pmatrix}
}_{\varepsilon_t}
\end{align*}
Note that as the diagonal elements of \(B_{0}\) equal unity, we don't assume that \(E(\varepsilon_t \varepsilon_t')=\Sigma_{\varepsilon} \) is the identity matrix.
So this is a different normalization rule as before.

Economically, the above restrictions embody to some extent a baseline IS-LM model:

\begin{enumerate}

\item The first equation provides three restrictions by assuming that the price level is predetermined
except that producers can respond immediately to aggregate supply shocks (e.g.\ an unexpected increase in oil or gas prices).
This is basically a horizontal aggregate supply (AS) curve; that is why we label the shock with AS\@.

\item The second equation provides no restrictions as it is assumed that real output responds to all other model variables contemporaneously.
This equation can be interpreted as an aggregate demand curve, or an IS curve, that is why we label the shock IS\@.

\item The third equation provides two restrictions by assuming that the interest rate does not react contemporaneously to aggregate measures of output and prices.
This represents a simple money supply function, according to which the central bank adjusts the rate of interest in relation to the money stock
and does not immediately observe aggregate output and aggregate prices.
Of course, this is in contrast to modern monetary policy theory (and practice).

\item The fourth equation provides one additional restriction by assuming that the first two entries in the last row are identical.
The underlying idea is that this equation represents a money demand function in which short-run money holdings rise in proportion to NOMINAL GNP\@.
Moreover, money holdings are allowed to be dependent on the interest rate.

\end{enumerate}
In sum, we have \(3+0+2+1=6\) restrictions, that is we have an exactly identified SVAR model,
identified by short-run exclusion restrictions on \(B_{0}\).
As the structure is not recursive, a Cholesky decomposition is not valid and hence we need to rely on a numerical optimizer (or an algorithm for short-run exclusion restrictions).

\item The code might look like this:
\lstinputlisting[style=Matlab-editor,basicstyle=\mlttfamily,title=\lstname]{progs/matlab/keatingSR.m}
Here is the helper function to impose the restrictions:
\lstinputlisting[style=Matlab-editor,basicstyle=\mlttfamily,title=\lstname]{progs/matlab/keatingSR_f.m}

\item An unexpected upward shift of the aggregate supply curve
\begin{itemize}
\item raises the price deflator
\item lowers real GNP
\item raises the federal funds rate
\item lowers the money supply at first, but ultimately raises M1
\end{itemize}
This response is not really appropriate for the U.S. economy during the considered sample period,
  making the identifying restrictions rather questionable.
\end{enumerate}
\fi
\newpage
\end{solution}