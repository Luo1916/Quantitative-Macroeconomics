\section[Maximum Likelihood Estimation of Laplace AR{(p)}]{Maximum Likelihood Estimation of Laplace AR{(p)}\label{ex:MaximumLikelihoodEstimationLaPlaceARp}}
Consider the AR{(1)} model with constant
\begin{align*}
y_t = c + \phi y_{t-1} + u_t
\end{align*}
Assume that the error terms \(u_t\) are {i.i.d.}\ Laplace distributed with known density
\begin{align*}
f_{u_{t}}(u) = \frac{1}{2} \exp{\left( -|u|\right)}
\end{align*}
Note that for the above parametrization of the Laplace distribution
  we have that \(E(u_t)=0\) and \(Var(u_t)=2\),
  so we are only interested in estimating \(c\) and \(\phi \)
  and not the standard deviation of \(u_t\) as it is fixed to 2.
\begin{enumerate}
\item
Derive the log-likelihood function conditional on the first observation.
\item
Write a function that calculates the conditional log-likelihood of \(c\) and \(\phi \).
\item
Load the dataset given in the CSV file \texttt{LaPlace.csv}.
Numerically find the maximum likelihood estimates of \(c\) and \(\phi \)
  by minimizing the negative conditional log-likelihood function.
\item
Compare your results with the maximum likelihood estimate under the assumption of Gaussianity.
That is, redo the estimation by minimizing the negative Gaussian log-likelihood function
  using the dataset given in the CSV file \texttt{LaPlace.csv}.
\end{enumerate}

\paragraph{Readings}
\begin{itemize}
\item \textcite{Lutkepohl_2004_UnivariateTimeSeries}
\end{itemize}


\begin{solution}\textbf{Solution to \nameref{ex:MaximumLikelihoodEstimationLaPlaceARp}}
\ifDisplaySolutions%
\begin{enumerate}

\item
Computation of the conditional expectation and variance:
\begin{gather*}
E[y_{t}|y_{t-1}] = c + \phi y_{t-1}
\\
Var[y_{t}|y_{t-1}] = var(u_t) = 2
\end{gather*}
Hence the conditional density is
\begin{align*}
f_t(y_{t}|y_{t-1}; c, \phi) = \frac{1}{2} \cdot e^{-|y_{t} -(c + \phi y_{t-1})|} = \frac{1}{2} \cdot e^{-|u_t|}
\end{align*}
The conditional log-likelihood function is therefore given by
\begin{align*}
\log L(y_{2}, \dots, y_{T};c, \phi) =-(T-1) \cdot \log(2) - \sum_{t=2}^{T} |u_{t}|
\end{align*}

\item
\lstinputlisting[style=Matlab-editor,basicstyle=\mlttfamily,title=\lstname]{progs/matlab/logLikeARpLaplace.m}

\item
\lstinputlisting[style=Matlab-editor,basicstyle=\mlttfamily,title=\lstname]{progs/matlab/ARpMLLaPlace.m}
\lstinputlisting[style=Matlab-editor,basicstyle=\mlttfamily,title=\lstname]{progs/matlab/AR1MLLaPlace.m}

\item
Note that the values are very close to each other.
Maximizing the Gaussian likelihood, even though the underlying distribution is not Gaussian,
  is also known as pseudo-maximum likelihood or quasi-maximum likelihood.
It usually performs surprisingly well if you cannot pin down the underlying distribution.
\end{enumerate}
\fi
\newpage
\end{solution}