\begin{enumerate}

\item First, let's rewrite the equations in matrix form:
\begin{align*}
\underbrace{%
\begin{bmatrix}
1 & b_{12} & b_{13} & b_{14}\\
b_{21} & 1 & b_{23} & b_{24}\\
b_{31} & b_{32} & 1 & b_{34}\\
b_{41} & b_{42} & b_{43} & 1
\end{bmatrix}}_{B_{0}}
\begin{bmatrix}
\Delta gnp_t\\
\Delta i_t\\
i_t - \Delta p_t\\
\Delta m_t - \Delta p_t
\end{bmatrix}
&=
\begin{bmatrix}
\varepsilon_t^{AS}\\
\varepsilon_t^{MS}\\
\varepsilon_t^{MD}\\
\varepsilon_t^{IS}
\end{bmatrix}
\\
\begin{bmatrix}
\Delta gnp_t\\
\Delta i_t\\
i_t - \Delta p_t\\
\Delta m_t - \Delta p_t
\end{bmatrix}
&=
\underbrace{%
\begin{bmatrix}
b^*_{11} & b^*_{12} & b^*_{13} & b^*_{14}\\
b^*_{21} & b^*_{22} & b^*_{23} & b^*_{24}\\
b^*_{31} & b^*_{32} & b^*_{33} & b^*_{34}\\
b^*_{41} & b^*_{42} & b^*_{43} & b^*_{44}
\end{bmatrix}}_{B^{-1}_0}
\begin{bmatrix}
\varepsilon_t^{AS}\\
\varepsilon_t^{MS}\\
\varepsilon_t^{MD}\\
\varepsilon_t^{IS}
\end{bmatrix}
\end{align*}
The long-run multiplier matrix is given by:
\begin{align*}
\Theta(1) = {A(1)}^{-1} B^{-1}_0
\end{align*}

Now let's derive the restrictions on the impact matrix \(B_{0}^{-1}\):
\begin{itemize}
\item Money supply shocks do not have contemporaneous effects on output growth, i.e.\
\begin{align*}
\frac{\partial \Delta gnp_t}{\partial \varepsilon_t^{MS}}= b^*_{12} = 0
\end{align*}

\item Money demand shocks do not have contemporaneous effects on output growth, i.e.\
\begin{align*}
\frac{\partial \Delta gnp_t}{\partial \varepsilon_t^{MD}}= b^*_{13} = 0	
\end{align*}
\end{itemize}
Summarizing this yields two restrictions:
\begin{align*}
B_{0}^{-1} =
\begin{pmatrix}
* & 0 & 0 & *\\
* & * & * & *\\
* & * & * & *\\
* & * & * & *
\end{pmatrix}
\end{align*}

Next, let's derive the restrictions on the structural matrix \(B_{0}\):
\begin{itemize}
\item Monetary authority does not react contemporaneously to changes in the price level.
This can be computed directly from equation~\eqref{eq:MS}:
\begin{align*}
\frac{\partial \Delta i_t}{\partial \Delta p_t}= b_{23} + b_{24} = 0 \Leftrightarrow b_{23} = -b_{24}
\end{align*}
\end{itemize}
Summarizing this yields one restriction:
\begin{align*}
B_{0} =
\begin{pmatrix}
1 & * & * & *\\
* & 1 & -b_{24} & b_{24}\\
* & * & 1 & *\\
* & * & * & 1
\end{pmatrix}
\end{align*}

\begin{itemize}
\item Money supply shocks, money demand shocks and aggregate demand shocks do not have long-run effects on the log of real GNP.
\end{itemize}
The restrictions on the long-run multiplier matrix are thus:
\begin{align*}
\Theta(1) =
\begin{pmatrix}
* & 0 & 0 & 0\\
* & * & * & *\\
* & * & * & *\\
* & * & * & *
\end{pmatrix}
\end{align*}
This yields three restrictions.

In total we have \(2+1+3=6\) restrictions, which is equal to the required number of \(K(K-1)/2=6\) of an exactly identified SVAR model.

Lastly, we need to keep in mind that the variance of the structural shocks is not normalized:
\begin{align*}
\Sigma_{\varepsilon} =
\begin{pmatrix}
\sigma_{11} & 0 & 0 & 0\\
0 & \sigma_{22} & 0 & 0\\
0 & 0 & \sigma_{33} & 0\\
0 & 0 & 0 & \sigma_{44}
\end{pmatrix}
\end{align*}

\item[2/3/4]
Here is the helper function to impose the restrictions:
\lstinputlisting[style=Matlab-editor,basicstyle=\mlttfamily,title=\lstname]{progs/matlab/gali1992_f.m}
The main code might look like this:
\lstinputlisting[style=Matlab-editor,basicstyle=\mlttfamily,title=\lstname]{progs/matlab/gali1992.m}

\end{enumerate}