\begin{enumerate}

\item The parameter vector \(\theta = [\beta,\sigma^2]'\) is a random variable with a probability distribution.
A Bayesian estimation of this distribution combines prior beliefs and information from the data:
\begin{enumerate}
\item Prior distribution \(p(\beta,\sigma^2)\)
\item Likelihood \(p(Y|\beta,\sigma^2)\)
\item Bayes' rule gives the joint posterior distribution \(p(\beta,\sigma^2|Y)\)
\end{enumerate}
Some terminology and useful relationships:
\begin{itemize}
\item joint posterior distribution of \(\beta \) and \(\sigma^2\):
\begin{align*}
p(\beta,\sigma^2|Y) = \frac{p(Y|\beta,\sigma^2) p(\beta,\sigma^2)}{p(Y)} \propto p(Y|\beta,\sigma^2) p(\beta,\sigma^2)
\end{align*}
\item marginal posterior distributions of \(\beta \) and \(\sigma^2\):
\begin{align*}
p(\beta|Y) &= \int_0^\infty p(\beta,\sigma^2|Y) d\sigma^2
\propto \int_0^\infty p(Y|\beta,\sigma^2) p(\beta,\sigma^2) d\sigma^2
\\
p(\sigma^2|Y) &= \int_{-\infty}^{\infty} p(\beta,\sigma^2|Y) d\beta
\propto \int_{-\infty}^{\infty} p(Y|\beta,\sigma^2) p(\beta,\sigma^2) d\beta
\end{align*}
Note that the first integral starts at zero because \(\sigma^2\) is non-negative.
\item conditional posterior distribution of \(\beta \) given \(\sigma^2\):
\begin{align*}
p(\beta|\sigma^2,Y) = \frac{p(\beta,\sigma^2|Y)}{p(\sigma^2|Y)}
\propto p(Y|\beta,\sigma^2) p(\beta,\sigma^2)
\end{align*}
Lastly, the following relationship is useful for simulations:
\begin{align*}
p(\beta,\sigma^2|Y) = p(\beta|\sigma^2,Y) p(\sigma^2|Y)
\end{align*}
\end{itemize}

\item Because \(u\) is normally distributed, \(Y\) is also Gaussian;
  hence, we can derive the precise form of the likelihood function:
\begin{align*}
p(Y|\beta,\sigma^2) = \frac{1}{{(2\pi \sigma^2)}^{T/2}} e^{-\frac{1}{2\sigma^2} (Y-X\beta)'(Y-X\beta)}
\end{align*}

\item First, assuming that \(\sigma^2\) is known and the prior for \(\beta \) is Gaussian,
  we have \(p(\beta) \sim N(\beta_0,\Sigma_0)\); that is, the prior density is given by:
\begin{align*} 
p(\beta) &= {(2\pi)}^{-K/2} |\Sigma_0|^{-1/2} e^{-\frac{1}{2} (\beta - \beta_0)' \Sigma_0^{-1} (\beta - \beta_0)}
\end{align*}
Note that the prior for \(\beta \) is independent of \(\sigma^2\), so we can also write:
\begin{align*}
p(\beta) = p(\beta|\sigma^2)
\end{align*}
Second, conditional on \(\sigma^2\) the likelihood is proportional to:
\begin{align*}
p(Y|\beta,\sigma^2) \propto e^{-\frac{1}{2\sigma^2} (Y-X\beta)'(Y-X\beta)}
\end{align*}
Third, combining conditional prior and likelihood yields the conditional posterior:
\begin{align*}
p(\beta|\sigma^2,Y) \propto e^{-\frac{1}{2} (\beta-\beta_0)' \Sigma_0^{-1} (\beta-\beta_0) - \frac{1}{2\sigma^2} (Y-X\beta)'(Y-X\beta)}
\end{align*}
One can show (see the references for the somewhat tedious algebra) that this is a Gaussian distribution
\begin{align*}
p(\beta|\sigma^2,Y) \sim N(\beta_1,\Sigma_1)
\end{align*}
with
\begin{align*}
\beta_1 &= {\left( \Sigma_0^{-1} + \sigma^{-2} (X'X) \right)}^{-1} \left( \Sigma_0^{-1} \beta_0 + \sigma^{-2} (X'Y) \right)
\\
\Sigma_1 &= {\left( \Sigma_0^{-1} + \sigma^{-2} (X'X) \right)}^{-1}
\end{align*}
Note that this is a closed-form expression for the conditional posterior of \(\beta \) given \(\sigma^2\).
Additionally, it is easy to sample from this conditional posterior,
  because it is a multivariate Gaussian distribution (e.g.\ using \texttt{mvnrnd} in MATLAB).

\item Assuming that \(\beta \) is known and the prior for the precision \(\frac{1}{\sigma^2}\) is Gamma,
  we have that \(p\left(\frac{1}{\sigma^2}\right) = p\left(\frac{1}{\sigma^2}|\beta\right) \sim \Gamma(s_0,v_0)\).\footnote{%
  Alternatively, one can assume an inverse Gamma distribution for \(\sigma \) or \(\sigma^2 \).}
That is, the prior density is proportional to:
\begin{align*}
p\left(\frac{1}{\sigma^2}|\beta\right) & \propto {\left(\frac{1}{\sigma^2} \right)}^{s_0-1} e^{-\frac{1}{v_0\sigma^2}}
\end{align*}
Conditional on \(\beta \) the likelihood is proportional to:
\begin{align*}
p(Y|\sigma^2,\beta) \propto {(\sigma^2)}^{-T/2} e^{-\frac{1}{2\sigma^2} (Y-X\beta)'(Y-X\beta)}
\end{align*}
Combining prior and likelihood yields (see the readings for the somewhat tedious algebra):
\begin{align*}
p\left(\frac{1}{\sigma^2}|\beta,Y\right) \sim \Gamma(s_1,v_1)
\end{align*}
where
\begin{align*}
s_1 &= s_0 + T
\\
v_1 &= v_0 + (Y-X\beta)'(Y-X\beta)
\end{align*}
Note that this is a closed-form expression for the conditional posterior of \(\frac{1}{\sigma^2}\) given \(\beta \).
Additionally, it is easy to sample from this conditional posterior,
  because it is a Gamma distribution (e.g.\ using \texttt{gamrnd} in MATLAB).

\item In the previous exercises we have derived the conditional posteriors in closed-form.
When both \(\beta \) and \(\sigma^2\) are unknown,
  we can specify the \textbf{joint prior} distribution for these parameters assuming a Gamma distribution for the marginal prior for \(\frac{1}{\sigma^2}\)
  and a normal distribution for the conditional prior for \(\beta|\frac{1}{\sigma^2}\),
  such that the \textbf{joint prior} is called \emph{Normal-Gamma}:
\begin{align*}
p\left(\beta, \frac{1}{\sigma^2}\right) = \underbrace{p(\beta|1/\sigma^2)}_{\text{Normal}} \underbrace{p\left(1/\sigma^2\right)}_{\text{Gamma}}
\end{align*}
The \textbf{joint posterior} density is given by the factorization of the conditional posteriors:
\begin{align*}
p\left(\beta,\frac{1}{\sigma^2}|Y\right) = p\left(\beta|\frac{1}{\sigma^2},Y\right) p\left(\frac{1}{\sigma^2}|Y\right)
\end{align*}
To make inference on \(\beta \), however, we need to know the marginal posterior:
\begin{align*}
p(\beta|Y) = \int_0^\infty p(\beta,1/\sigma^2|Y) d(1/\sigma^2)
\end{align*}
This integration is very hard, but we can make use of a numerical Monte Carlo integration approach.

\paragraph{Gibbs sampling}
The idea of Gibbs sampling is to repeatedly sample from the conditional posterior distributions (derived in {3.} and {4.})
  to get an approximation of the marginal and joint posterior distributions of the parameters.

Basic steps of the Gibbs sampling algorithm:
\begin{itemize}
\item Set priors and initial guess for \(\sigma^2\)
\item Sample \(\beta \) conditional on \(1/\sigma^2\) (see {3.})
\item Sample \(1/\sigma^2\) conditional on \(\beta \) (see {4.})
\item Repeat (2) and (3) a large number of times \(R\) and keep the last \(L\) draws.
\item Use the \(L\) draws to make inference on \(\beta \) and \(\sigma \).
\end{itemize}

\end{enumerate}