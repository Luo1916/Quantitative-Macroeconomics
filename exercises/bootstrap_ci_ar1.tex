\section[Bootstrap Confidence Interval For AR{(1)} Coefficient]{Bootstrap Confidence Interval For AR{(1)} Coefficient\label{ex:BootstrapConfidenceIntervalARone}}
Consider the AR{(1)} model with constant
\begin{equation*}
y_{t}=c +\phi y_{t-1}+u_{t}
\end{equation*}
for \(t=1,\ldots ,T\) with iid error terms \(u_{t}\) and \(E(u_{t}|y_{t-1})=0\).
Usually, a \( (1-\alpha)\% \) confidence interval for \( \phi \) is constructed using the normal (or Student's t) approximation:
\begin{gather*}
\left[ \hat{\phi}-z_{\alpha/2}\cdot SE(\hat{\phi});\ \hat{\phi}+z_{1-\alpha/2}\cdot SE(\hat{\phi})\right]
\end{gather*}
with \(\hat{\phi}\) denoting the OLS estimate,
  \(SE(\hat{\phi})\) the estimated standard error of \( \phi \)
  and \(z_{\alpha/2}\) the \(\alpha/2\) quantile of the standard normal distribution (or t-distribution).
If one does not know the asymptotic distribution of a test statistic
  (or it has a very complicated form),
  one often relies on a nonparametric simulation approach.
To this end, we are going to do a so-called \enquote{\emph{bootstrap}},
  i.e.\ we recompute the t-statistics a large number of times on artificial data
  generated from resampled residuals.

\begin{enumerate}
\item
What is a \enquote{Bootstrap approximation}?
Provide insight into the basic idea and possible applications of this statistical technique.

\item
Write a program for the following:
\begin{itemize}
  \item Simulate \(T=100\) observations with \(c=1\), \(\phi=0.8\)
    and errors drawn from e.g.\ the exponential distribution such that \(E(u_t)=0\).
  \item Estimate the model with OLS and calculate the t-statistic
    \(\tau=\frac{\hat{\phi}}{SE(\hat{\phi})}\). 
  \item Store the OLS residuals in a vector
    \(\hat{u} = (\hat{u}_{2},\ldots ,\hat{u}_{T})'\).
  \item Set \(B=10000\) and initialize the output vector
    \(\tau^{\ast} = (\tau_1^\ast,\ldots,\tau_B^\ast)\).
  \item For \( b=1,\ldots,B \):
  \begin{itemize}
    \item Draw a sample \textbf{with replacement} from \(\hat{u}\)
      and save it as \(u^{\ast} = u_{2}^{\ast},\ldots ,u_{T}^{\ast }\).
    \item Initialize an artificial time series \( y_t^\ast \) with \(T\) observations
      and set \(y_1^\ast = y_1\).
    \item For \(t=2,\ldots ,T\) generate
    \begin{align*}
    y_{t}^{\ast }=\hat{c}+\hat{\phi}y^\ast_{t-1}+u_{t}^{\ast }
    \end{align*}
    \item On this artificial dataset estimate an AR{(1)} model.
    Denote the estimated OLS coefficient \(\phi^\ast \)
      and corresponding estimated standard deviation \(SE(\phi^\ast)\).
    Store the following t-statistic in your output vector at position \(b\):
    \begin{align*}
    \tau^\ast = \frac{\phi^\ast - \hat{\phi}}{SE(\phi^\ast)}
    \end{align*}
    \end{itemize}
    \item Sort the output vector such that \(\tau_{(1)}^\ast \leq \cdots \leq \tau_{(B)}^\ast \).
    \item The \enquote{\emph{bootstrap approximate}} confidence interval for \(\phi \) is then given by
    \begin{align*}
    \left[ \hat{\phi}-\tau_{((1-\alpha /2)B)}^{\ast }\cdot SE(\hat{\phi});\ \hat{\phi}-\tau_{((\alpha/2)B)}^{\ast }\cdot SE(\hat{\phi})\right] 
    \end{align*}
    Set \(\alpha=0.05\) and compare this with the normal approximation.
    \item Redo the exercise for \(T=30\) and \(T=10000\). Comment on your findings.
\end{itemize}

\end{enumerate}

\paragraph{Readings}
\begin{itemize}
\item \textcite[Ch. 12]{Kilian.Lutkepohl_2017_StructuralVectorAutoregressive}
\end{itemize}

\begin{solution}\textbf{Solution to \nameref{ex:BootstrapConfidenceIntervalARone}}
\ifDisplaySolutions%
\lstinputlisting[style=Matlab-editor,basicstyle=\mlttfamily,title=\lstname]{progs/matlab/bootstrapCIAR1.m}
For large \(T\) the bootstrap CI are almost identical to the asymptotic CIs, for small \(T\) they are narrower.

No efficiency gain from imposing parametric assumptions even when that assumption is true,
  whereas imposing the wrong parametric structure tends to undermine the accuracy of the bootstrap inference.

\(\hookrightarrow \) Nonparametric approach is typically strictly preferred in practice.

\fi
\newpage
\end{solution}