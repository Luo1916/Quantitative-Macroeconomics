\section[Central Limit Theorem For Dependent Data]{Central Limit Theorem For Dependent Data\label{ex:CentralLimitTheoremDependentData}}
Suppose that the sequence \(Y_{1},Y_{2},\ldots \) is an AR{(1)} process, i.e.\
\begin{align*}
Y_{t}-\mu =\phi \left(Y_{t-1}-\mu\right) +\varepsilon _{t}
\end{align*}
where \(\varepsilon _{t}\sim iid(0,\sigma _{\varepsilon }^{2})\) is
(not necessarily but in our case) normally distributed and \(|\phi |<1\).

\begin{enumerate}
\item
Briefly state and describe the intuition of the \enquote{Lindeberg-Levy Central Limit Theorem} for iid random variables.
What does \enquote{convergence in distribution} mean?
Why can we not use the theorem for the AR{(1)} process?

\item
Show that \(Y_t\) has mean equal to \(\mu \) and finite variance equal to \(\sigma_\varepsilon^2/(1-\phi^2)\).

\item To derive the asymptotic distribution of the sample mean, do the following steps:
\begin{enumerate}
  \item
  Derive the asymptotic distribution of \(\frac{1}{\sqrt{T} } \sum_{t=1}^T \varepsilon_t\).

  \item
   Show that
  \begin{align*}
  \frac{1}{\sqrt{T}} \sum_{t=1}^T \varepsilon_t = \sqrt{T}\left[(1-\phi)\left(\hat{\mu}-\mu\right) + \phi\left(\frac{Y_T - Y_0}{T}\right)\right]
  \end{align*}
  with \(\hat{\mu} =\frac{1}{T}\sum_{t=1}^{T}Y_{t}\).

  \item
  Show that
  \begin{align*}
  \textsl{plim}\left[\frac{\phi}{1-\phi}\left(\frac{Y_T - Y_0}{\sqrt{T}}\right)\right] = 0
  \end{align*}
  \\
  \emph{Hint: Use Tchebychev's Inequality,
  i.e.\ for a random variable \(X\) with expectation \(\mu_x\)
  and finite variance \(\sigma_x^2\):}
  \begin{align*}
  \Pr(|X-\mu_x|> \delta) \leq \frac{\sigma_x^2}{\delta^2}
  \end{align*}
  \emph{for any small real number \(\delta>0\).}

  \item
  Put your results of (a), (b) and (c) together and derive the asymptotic distribution of the sample mean.
  That is, show that
  \begin{align*}
  Z_{T} =\sqrt{T}\frac{\hat{\mu} -\mu }{\sigma_Z} \overset{d}{\rightarrow} U \sim N(0,1)
  \end{align*}
  for \(\sigma_Z=\sqrt{\sigma_\varepsilon^2/{(1-\phi)}^2}\).
\end{enumerate}

\item
 Write a program to demonstrate the central limit theorem for the AR{(1)} process. To this end:

\begin{itemize}

  \item
  Simulate \(B=5000\) stationary (e.g.\
  \(\phi=0.8\)) AR{(1)} processes with each \(T=10000\) observations.
  Store these in a \(T \times B\) matrix \(Y\).

  \item
  Compute \(\hat{\mu}\) for each column of \(Y\).

  \item
  Plot the histograms of the standardized variables according to the Lindeberg-Levy Central Limit Theorem:
  \begin{align*}
  \widetilde{Z}_T = \sqrt{T}\frac{\hat{\mu}-\mu}{\sigma_{\varepsilon }/\sqrt{1-\phi^2}}
  \end{align*}
  and of the correct standardized variables that we derived in 3(d):
  \begin{align*}
  Z_T = \sqrt{T}\frac{\hat{\mu}-\mu}{\sigma_{\varepsilon }/(1-\phi)}
  \end{align*}
  Compare the histograms to the standard normal distribution.

\end{itemize}

\end{enumerate}

\paragraph{Readings}
\begin{itemize}
\item \textcite{Crack.Ledoit_2010_CentralLimitTheorems}
\item \textcite[App. C]{Lutkepohl_2005_NewIntroductionMultiple}
\item \textcite[App. C]{Neusser_2016_TimeSeriesEconometrics}
\item \textcite[Ch. 5]{White_2001_AsymptoticTheoryEconometricians}
\end{itemize}

\begin{solution}\textbf{Solution to \nameref{ex:CentralLimitTheoremDependentData}}
\ifDisplaySolutions%
\input{exercises/central_limit_theorem_dependent_data_solution.tex}
\fi
\newpage
\end{solution}