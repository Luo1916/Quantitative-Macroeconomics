\section[Maximum Likelihood Estimation of Gaussian AR{(p)}]{Maximum Likelihood Estimation of Gaussian AR{(p)}\label{ex:MaximumLikelihoodEstimationGaussianARp}}
Consider an AR{(p)} model with a constant and linear trend:
\begin{align*}
y_t = c + d\cdot t + \phi_1 y_{t-1} + \cdots + \phi_p y_{t-p} +u_{t}=Y_{t-1}\theta + u_t
\end{align*}
  where \(Y_{t-1}=(1,t, y_{t-1},\ldots,y_{t-p})\) is the matrix of regressors,
  \(\theta = (c,d,\phi_1,\ldots,\phi_p)\) the parameter vector
  and the error terms \(u_t\) are white noise and normally distributed, i.e.\
  \( u_t\sim N(0,\sigma_u^2) \) and \(E[u_t u_s]=0\) for \(t\neq s\).
If the sample distribution is known to have probability density function \(f(y_1,\ldots,y_T)\),
  an estimation with Maximum Likelihood (ML) is possible.
To this end, we decompose the joint distribution by
\begin{align*}
f(y_1,\ldots,y_T|\theta,\sigma_u^2)= f_1(y_1|\theta,\sigma_u^2) \times f_2(y_2|y_1,\theta,\sigma_u^2)\times \cdots \times f_T(y_T|y_{T-1},\ldots,y_1,\theta,\sigma_u^2)
\end{align*}
Then the log-likelihood is
\begin{align*}
\log f(y_1,\ldots,y_T|\theta,\sigma_u^2)=\sum_{t=1}^T \log f_t(y_t|y_{t-1},\ldots,y_1,\theta,\sigma_u^2)
\end{align*}
Let's denote the values that maximize the log-likelihood as \(\tilde{\theta}\) and \(\tilde{\sigma}_u^2\).
ML estimators have (under general assumptions) an asymptotic normal distribution
\begin{align*}
\sqrt{T}\begin{pmatrix}\tilde{\theta}-\theta \\ \tilde{\sigma}^2_u - \sigma_u^2 \end{pmatrix} \overset{d}{\rightarrow} U \sim N(0,I_a{(\theta,\sigma_u^2)}^{-1})
\end{align*}
where \(I_a(\theta,\sigma_u)\) is the asymptotic information matrix.
Recall that the asymptotic information matrix is the limit of minus the expectation of the Hessian of the log-likelihood divided by the sample size.
\begin{align*}
I_a(\theta,\sigma_u^2) = \lim\limits_{T\rightarrow \infty}-\frac{1}{T} E
\begin{pmatrix}
\frac{\partial^2 \log l}{\partial \theta^2} & \frac{\partial^2 \log l}{\partial \theta  \partial \sigma_u^2}
\\
\frac{\partial^2 \log l}{\partial \sigma_u^2  \partial \theta} & \frac{\partial^2 \log l}{\partial {(\sigma_u^2)}^2}  
\end{pmatrix}
\end{align*}

\begin{enumerate}
\item
First consider the case of \(p=1\)
\begin{enumerate}
  \item Derive the exact log-likelihood function for the \(AR(1)\) model with \(|\theta|<1\) and \(d=0\):
  \begin{align*}
  y_t = c + \theta y_{t-1} + u_t
  \end{align*}
  \item Why do we often look at the log-likelihood function instead of the actual likelihood function?
  \item Regard the value of the first observation as deterministic or, equivalently,
    note that its contribution to the log-likelihood disappears asymptotically.
    Maximize analytically the conditional log-likelihood to get the ML estimators for \(\theta \) and \(\sigma_u\).
    Compare these to the OLS estimators.
\end{enumerate}
\item
Now consider the general \(AR(p)\) model.
\begin{enumerate}
  \item Write a function \texttt{logLikeARpNorm{(\(x\),\(y\),\(p\),\(const\))}}
    that computes the log-likelihood value
    conditional on the first \(p\) observations of a Gaussian \(AR(p)\) model:
  \begin{align*}
  \log l(\theta,\sigma_u)= -\frac{T-p}{2}\log(2\pi)-\frac{T-p}{2}\log(\sigma_u^2)-\sum_{t=p+1}^{T}\frac{u_t^2}{2\sigma_u^2}
  \end{align*}
  where \(x=(\theta',\sigma_u)'\), \(y\) denotes the data vector,
  \(p\) the number of lags and \(const\) is equal to 1 if there is a constant,
  and equal to 2 if there are both a constant and linear trend in the model.
  \item Write a \texttt{function ML = ARpML{(\(y\),\(p\),\(const\),\( \alpha \))}}
    that takes as inputs a data vector \(y\), number of lags \(p\)
    and \(const=1\) if the model has a constant term
    or \(const=2\) if the model has both a constant term and linear trend.
  \(\alpha \) denotes the significance level.
  The function computes 
  \begin{itemize}
    \item the maximum likelihood estimates of an AR{(p)} model
      by numerically minimizing the negative conditional log-likelihood function using e.g.\ \texttt{fminunc}
    \item the standard errors by means of the asymptotic covariance matrix, i.e.\
      the inverse of the hessian of the negative log-likelihood function
      (hint: gradient-based optimizers also output the hessian)
  \end{itemize}
  Save all results into a structure \enquote{ML} containing the estimates of \(\theta \),
    its standard errors, t-statistics and p-values as well as the ML estimate of \(\sigma_u\).
    
  \item Load simulated data given in the CSV file \texttt{AR4.csv} and estimate an AR{(4)} model with constant term.
  Compare your results with the OLS estimators from the previous exercise.
\end{enumerate}
\end{enumerate}

\paragraph{Readings}
\begin{itemize}
\item \textcite{Lutkepohl_2004_UnivariateTimeSeries}
\end{itemize}


\begin{solution}\textbf{Solution to \nameref{ex:MaximumLikelihoodEstimationGaussianARp}}
\ifDisplaySolutions%
\begin{enumerate}

\item
Let's first consider the AR{(1)} model
\begin{enumerate}
  \item
  The first observation \(y_1\) is a random variable with mean and variance equal to:
  \begin{align*}
  E[y_1] = \mu = \frac{c}{1-\theta} \quad\text{and}\quad Var[y_1] = \frac{\sigma_u^2}{1-\theta^2}
  \end{align*}
  Since the errors are Gaussian, \(y_1\) is also Gaussian, i.e.\
  \(y_1 \sim N\left(\frac{c}{1-\theta},\frac{\sigma_u^2}{1-\theta^2}\right)\).
  The pdf is thus:
  \begin{align*}
  f_1(y_1|\theta,\sigma_u^2) = \frac{1}{\sqrt{2\pi}\sqrt{\sigma_u^2/(1-\theta^2)}}\exp\left \{-\frac{1}{2}\frac{{[y_1-(c/(1-\theta))]}^2}{\sigma_u^2/(1-\theta^2)}\right \}
  \end{align*}
  The second observation \(y_2\) conditional on \(y_1\) is given by \(y_2 = c + \theta y_1 + u_2\).
  Conditional on \(y_1\), \(y_2\) is thus the sum of a deterministic term (\(c+\theta y_1\))
    and the \(N(0,\sigma_u^2)\) variable \(u_2\).
  Hence:
  \begin{align*}
  y_2|y_1 \sim N(c+\theta y_1,\sigma_u^2)
  \end{align*}
  and the pdf is given by:
  \begin{align*}
  f_2(y_2|y_1,\theta,\sigma_u^2) = \frac{1}{\sqrt{2\pi\sigma_u^2}}\exp\left \{-\frac{1}{2}\frac{[y_2-c-\theta y_1]^2}{\sigma_u^2}\right \}
  \end{align*}
  The joint density of observations 1 and 2 is then just:
  \begin{align*}
  f(y_2,y_1|\theta,\sigma_u^2) = f_2(y_2|y_1,\theta,\sigma_u^2) \cdot f_1(y_1|\theta,\sigma_u^2)
  \end{align*}
  In general the values of \(y_1,y_2,\ldots ,y_{t-1}\) matter for \(y_t\) only through the value \(y_{t-1}\)
    and the density of observation \(t\) conditional on the preceding \(t-1\) observations is given by
  \begin{align*}
  f_t(y_t|y_{t-1},\theta,\sigma_u^2) = \frac{1}{\sqrt{2\pi\sigma_u^2}}\exp\left \{-\frac{1}{2}\frac{[y_t-c-\theta y_{t-1}]^2}{\sigma_u^2}\right \}
  \end{align*}
  The likelihood of the complete sample can thus be calculated as:
  \begin{align*}
  f(y_T,y_{T-1},\ldots ,y_1|\theta,\sigma_u^2)=f_1(y_1|\theta,\sigma_u^2)\cdot \prod_{t=2}^{T}f_t(y_t|y_{t-1},\theta,\sigma_u^2)
  \end{align*}
  The log-likelihood is therefore
  \begin{align*}
  \log l(\theta,\sigma_u^2)&= \log f_1(y_1|\theta,\sigma_u^2)+  \sum_{t=2}^{T}\log f_t(y_t|y_{t-1},\theta,\sigma_u^2)
  \\
  &= -\frac{1}{2}\log(2\pi) -\frac{1}{2}\log(\sigma_u^2/(1-\theta^2))-\frac{(y_1-{(c/(1-\theta))}^2)}{2\sigma_u^2/(1-\theta^2)}
  \\
  &-((T-1)/2)\log(2\pi)-((T-1)/2)\log(\sigma_u^2)-\sum_{t=2}^{T}\frac{{(y_t-c-\theta y_{t-1})}^2}{2\sigma_u^2}
  \end{align*}

  \item
  Theoretically it does not matter whether we consider the log-likelihood or the actual likelihood function,
    as the value that maximizes the likelihood also maximize the log-likelihood,
    because the log is a monotone transformation.
  However, it is usually easier to work with sums instead of products theoretically,
    e.g.\ the LLN or CT are based on sums.
  Computationally, working with products is typically impossible
    as the resulting values very quickly surpass machine precision (they quickly go to \(\pm \infty \));
    working with sums does not have this problem.
  So from a computational perspective we will exclusively work with the log-likelihood function.

  \item
  Discarding the first observation, the conditional log-likelihood is given by:
  \begin{align*}
  \log l^c(\theta,\sigma_u^2)&= -((T-1)/2)\log(2\pi)-((T-1)/2)\log(\sigma^2_u)-\sum_{t=2}^{T}\frac{{(y_t-c-\theta y_{t-1})}^2}{2\sigma_u^2}
  \\
  &= -((T-1)/2)\log(2\pi)-((T-1)/2)\log(\sigma^2_u)-\sum_{t=2}^{T}\frac{u_t^2}{2\sigma_u^2}
  \end{align*}
  Note that the first two sums do not depend on \(\theta \);
    thus, when maximizing \(\log l^c(\theta,\sigma_u^2)\) with respect to \(\theta \),
    we are basically minimizing the squared residuals,
    which will simply yield the OLS estimator.
  The estimator for the variance, however, is different,
    as we are dividing the sum of squared residuals (\(\sum_{t=2}^T u_t^2\))
    by \(T^{eff}=(T-1)\) when doing {ML} instead of by \(T^{eff}-1\) when doing {OLS}.
  Obviously, for large \(T\) this does not matter.
\end{enumerate}

\item
\lstinputlisting[style=Matlab-editor,basicstyle=\mlttfamily,title=\lstname]{progs/matlab/logLikeARpNorm.m}

\item
\lstinputlisting[style=Matlab-editor,basicstyle=\mlttfamily,title=\lstname]{progs/matlab/ARpML.m}

\item
\lstinputlisting[style=Matlab-editor,basicstyle=\mlttfamily,title=\lstname]{progs/matlab/AR4ML.m}
The estimates for the coefficients are the same,
  but slightly different for the standard deviation of the error term.
\end{enumerate}
\fi
\newpage
\end{solution}